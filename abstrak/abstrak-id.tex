\chapter*{ABSTRAK}

\addcontentsline{toc}{chapter}{ABSTRAK}

\vspace{2ex}

\begin{center}
  \large\textbf{\tatitle{}}
\end{center}

\vspace{2ex}

\begingroup
% Menghilangkan padding
\setlength{\tabcolsep}{0pt}

\noindent
\begin{tabularx}{\textwidth}{l >{\centering}m{2em} X}
  \textbf{Nama / NRP}           & : & \textbf{\name{} / \nrp{}} \\
  \textbf{Departemen}           & : & \textbf{\department{}}    \\
  \textbf{Dosen Pembimbing I}   & : & \textbf{\advisor{}}       \\
  \textbf{Dosen Pembimbing II}  & : & \textbf{\coadvisor{}}     \\
\end{tabularx}
\endgroup

\noindent
\textbf{Abstrak}

% Ubah paragraf berikut dengan abstrak dari tugas akhir
Sumber daya komputasi tidak selalu digunakan.
Sumber daya komputasi terkadang berada pada posisi
\emph{idle} dan dapat sewaktu-waktu digunakan kembali.

% Pada penelitian ini kami mengajukan sistem \emph{provisioning}
% \emph{virtual cluster} yang dapat diciptakan sewaktu-waktu.

\vspace{2ex}
% Ubah kata-kata berikut dengan kata kunci dari tugas akhir
\noindent
\textbf{Kata Kunci: \emph{Multi-Tenancy}, \emph{Virtual Machine}, Kubernetes, \emph{Virtual Cluster}.}
