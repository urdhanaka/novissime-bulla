\chapter*{ABSTRAK}

\addcontentsline{toc}{chapter}{ABSTRAK}

\vspace{2ex}

\begin{center}
  \large\textbf{\tatitle{}}
\end{center}

\vspace{2ex}

\begingroup
% Menghilangkan padding
\setlength{\tabcolsep}{0pt}

\noindent
\begin{tabularx}{\textwidth}{l >{\centering}m{2em} X}
  \textbf{Nama / NRP}           & : & \textbf{\name{} / \nrp{}} \\
  \textbf{Departemen}           & : & \textbf{\department{}}    \\
  \textbf{Dosen Pembimbing I}   & : & \textbf{\advisor{}}       \\
  \textbf{Dosen Pembimbing II}  & : & \textbf{\coadvisor{}}     \\
\end{tabularx}
\endgroup

\noindent
\textbf{Abstrak}

% Ubah paragraf berikut dengan abstrak dari tugas akhir
Sumber daya komputasi seperti komputer tidak selalu berada
pada posisi digunakan atau berada pada posisi \emph{idle}.
Untuk menghindari pemborosan sumber daya komputasi tersebut,
dibutuhkan mekanisme untuk menyewakan sumber daya yang \emph{idle}
ke pengguna yang membutuhkan. Salah satu \emph{tool} yang
dapat digunakan adalah Kubernetes. Namun, penggunaan Kubernetes
pada skala besar membutuhkan sumber daya yang \emph{fix}. Pada
penelitian kali ini, kami mengajukan sistem \emph{provisioning}
klaster Kubernetes dalam bentuk \emph{virtual cluster}. \emph{Virtual cluster}
merupakan pendekatan yang memungkinkan sumber daya komputasi disediakan secara
\emph{on-demand}. Selain itu, \emph{multi-tenancy} juga akan diimplementasikan
agar bisa digunakan oleh banyak pengguna. Hasil dari penelitian ini
adalah sistem \emph{provisioning virtual cluster} dapat digunakan sebagai
sumber daya komputasi tambahan untuk penggunanya. Selain itu, komputer fisik
untuk \emph{provisioning} dapat melayani lebih dari satu \emph{virtual cluster}

\vspace{2ex}
% Ubah kata-kata berikut dengan kata kunci dari tugas akhir
\noindent
\textbf{Kata Kunci: \emph{Multi-Tenancy}, \emph{Virtual Machine}, Kubernetes, \emph{Virtual Cluster}.}
