\chapter*{ABSTRACT}

\addcontentsline{toc}{chapter}{ABSTRACT}

\vspace{2ex}

\begin{center}
  \large\textbf{\engtatitle{}}
\end{center}

\vspace{2ex}

\begingroup
% Menghilangkan padding
\setlength{\tabcolsep}{0pt}

\noindent
\begin{tabularx}{\textwidth}{l >{\centering}m{3em} X}
  \textbf{Name / NRP} & : & \textbf{\name{} / \nrp{}} \\
  \textbf{Department} & : & \textbf{\engdepartment{}} \\
  \textbf{Advisor I}  & : & \textbf{\advisor{}}       \\
  \textbf{Advisor II} & : & \textbf{\coadvisor{}}     \\
\end{tabularx}
\endgroup

\noindent
\textbf{Abstract}

% Ubah paragraf berikut dengan abstrak dari tugas akhir dalam Bahasa Inggris
Computing resources such as computers are not always in use or in idle
position. This can lead to computing resources waste. To prevent that,
a mechanism is needed to utilize these resources  by renting out the
idle resources to users in need. One of the tools that can be used is Kubernetes.
However, using Kubernetes usually requires fix amount of resources. In this study,
we propose a Kubernetes provisioning system in the form of virtual cluster. Virtual
cluster is an approach that allows computing resources to be provided on-demand. In
addition, multi-tenancy will also be implemented so that computing resources that are
rented can be used by more than one user. The result of this study is that the virtual
cluster provisioning system can be used as additional computing resources for its users
on-demand. In addition, user also shares borrowed computing resources with other users.

\vspace{2ex}
% Ubah kata-kata berikut dengan kata kunci dari tugas akhir dalam Bahasa Inggris
\noindent
\textbf{Keywords \emph{Multi-Tenancy}, \emph{Virtual Machine}, Kubernetes, \emph{Virtual Cluster}.}
