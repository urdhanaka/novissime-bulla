% Atur variabel berikut sesuai namanya

% nama
\newcommand{\name}{Urdhanaka Aptanagi}
\newcommand{\authorname}{Aptanagi, Urdhanaka}
\newcommand{\nickname}{Elon}
\newcommand{\advisor}{Royyana Muslim Ijtihadie, S.Kom., M.Kom., Ph.D.}
\newcommand{\coadvisor}{Ary Mazharuddin Shiddiqi, S.Kom., M.Comp.Sc., Ph.D.}
\newcommand{\examinerone}{Dr. Galileo Galilei, S.T., M.Sc}
\newcommand{\examinertwo}{Friedrich Nietzsche, S.T., M.Sc}
\newcommand{\examinerthree}{Alan Turing, ST., MT}
\newcommand{\headofdepartment}{Ary Mazharuddin Shiddiqi, S.Kom., M.Comp.Sc., Ph.D.}

  % Dosen Pembimbing&:& 1. Royyana Muslim Ijtihadie, S.Kom., M.Kom., Ph.D.\\

% identitas
\newcommand{\nrp}{5025211123}
\newcommand{\advisornip}{19770824 200304 1 001}
\newcommand{\coadvisornip}{19810620 200501 1 003}
\newcommand{\examineronenip}{18560710 194301 1 001}
\newcommand{\examinertwonip}{18560710 194301 1 001}
\newcommand{\examinerthreenip}{18560710 194301 1 001}
\newcommand{\headofdepartmentnip}{19810620 200501 1 003}

% judul
\newcommand{\tatitle}{IMPLEMENTASI \emph{MULTI-TENANCY} UNTUK PROVISIONING KLASTER KUBERNETES}
\newcommand{\engtatitle}{\emph{MULTI-TENANCY IMPLEMENTATION FOR KUBERNETES CLUSTER PROVISIONING}}

% tempat
\newcommand{\place}{Surabaya}

% jurusan
\newcommand{\studyprogram}{Teknik Informatika}
\newcommand{\engstudyprogram}{Informatics}

% fakultas
\newcommand{\faculty}{Teknologi Elektro dan Informatika Cerdas}
\newcommand{\engfaculty}{Intelligent Electrical and Informatics Technology}

% singkatan fakultas
\newcommand{\facultyshort}{FTEIC}
\newcommand{\engfacultyshort}{ELECTICS}

% departemen
\newcommand{\department}{Teknik Informatika}
\newcommand{\engdepartment}{Informatics}

% kode mata kuliah
\newcommand{\coursecode}{EF234801}
