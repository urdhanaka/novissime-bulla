\chapter{PENDAHULUAN}
\label{chap:pendahuluan}

\section{Latar Belakang}
\label{sec:latarbelakang}

Organisasi menggunakan banyak \emph{tools} untuk mencapai
tujuannya. Salah satu \emph{tools} tersebut adalah sumber daya
komputasi berupa komputer yang tersebar dalam organisasi tersebut. Komputer-komputer
tersebut dapat digunakan untuk mencapai tujuan dari organisasi seperti 
melakukan komputasi data yang dimiliki oleh organisasi, pembuatan
surat pada organisasi, dan lain-lain.

Komputer-komputer organisasi tersebut berada pada posisi
\emph{idle} atau tidak terpakai pada jam di luar kerja, namun
komputer-komputer tersebut ada yang berada pada posisi \emph{idle}
dan tidak pada saat jam kerja. Sumber daya komputasi yang berada
pada posisi tidak terpakai tersebut merupakan pemborosan sumber
daya karena komputer tersebut tidak melakukan proses komputasi apapun.

Untuk menghindari pemborosan tersebut, diperlukan sebuah cara untuk
memanfaatkan komputer yang berada pada kondisi \emph{idle}. Salah satu
cara yang dapat digunakan adalah menggunakan mekanisme \emph{provisioning}
agar komputer yang berada pada kondisi \emph{idle} dapat dipakai oleh
pengguna. Namun, komputer tersebut dapat saja digunakan oleh pengguna lainnya.

Salah satu \emph{tools} yang dipakai untuk proses \emph{provisioning} adalah
Kubernetes. Kubernetes merupakan sebuah platform sumber terbuka untuk mengatur
layanan dan beban kerja yang dikemas (\emph{containerized}). Kubernetes akan
membentuk sebuah klaster yang terdiri dari satu atau lebih \emph{nodes} atau
komputer yang saling bekerja sama dalam menjalankan layanan atau beban kerja
yang diberikan kepada klaster tersebut. Penggunaan Kubernetes pada skala besar
membutuhkan sumber daya yang pasti.

Pada tugas akhir ini, akan dikembangkan sebuah sistem \emph{provisioning} untuk \emph{virtual cluster}.
\emph{Virtual cluster} adalah pendekatan yang memungkinkan sumber daya komputasi tersebut disediakan
secara \emph{on-demand} dan tidak menggunakan komputer atau \emph{node} fisik yang sama. \emph{Multi-tenancy}
akan diterapkan agar komputer yang digunakan untuk \emph{provisioning virtual cluster} dapat digunakan
oleh banyak pengguna.

\section{Rumusan Permasalahan}
\label{sec:rumusanpermasalahan}

Dari latar belakang tersebut maka dapat dijabarkan permasalahan sebagai berikut:

\begin{enumerate}[nolistsep]
  
  \item Bagaimana membentuk \emph{virtual cluster} Kubernetes dengan melakukan
    \emph{resource allocation} di atas \emph{resource} komputasi yang dinamis?

  \item Bagaimana cara mengimplementasikan \emph{multi-tenancy} pada proses \emph{provisioning}
    \emph{virtual cluster} Kubernetes?

  \item Bagaimana cara untuk mengevaluasi mekanisme \emph{provisioning virtual cluster}
    Kubernetes?

\end{enumerate}

\section{Batasan Masalah}
\label{sec:batasanmasalah}

Batasan-batasan dari penelitian atau implementasi adalah sebagai berikut:

\begin{enumerate}[nolistsep]

  \item Hasil implementasi berupa aplikasi \emph{provisioning} yang terdapat di setiap
    komputer yang akan digunakan dan aplikasi web sebagai \emph{dashboard}

  \item Kubernetes yang digunakan adalah K3s sebagai salah satu distribusi
    Kubernetes dengan versi v1.33.1+k3s1

  \item Aplikasi web berupa \emph{dashboard} dan aplikasi \emph{provisioning} dibuat
    menggunakan bahasa Golang versi 1.24.3

  \item Aplikasi \emph{provisioning} menggunakan libvirt dan libvirtd dengan versi 11.0.0

\end{enumerate}

\section{Tujuan}
\label{sec:Tujuan}

Tujuan dari implementasi ini adalah untuk mengembangkan sistem \emph{provisioning}
yang dapat membuat VM dan secara otomatis tergabung ke dalam sebuah \emph{virtual cluster}
Kubernetes secara dinamis sesuai dengan kriteria
\emph{resource} permintaan pengguna.

\section{Manfaat}
\label{sec:sistematikapenulisan}

Manfaat dari implementasi ini adalah untuk menyediakan sebuah sistem penyewaan
sumber daya komputasi dalam bentuk \emph{virtual cluster} yang dapat
diciptakan sewaktu-waktu dengan sumber daya yang sesuai dengan keinginan pengguna.
Diharapkan sistem penyewaan ini dapat digunakan sebagai sumber daya komputasi
tambahan yang dapat digunakan untuk proses komputasi dan untuk
meningkatkan utilisasi dari sumber daya komputasi yang dimiliki.
