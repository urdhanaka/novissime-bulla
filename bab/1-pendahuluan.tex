\chapter{PENDAHULUAN}
\label{chap:pendahuluan}

% Ubah bagian-bagian berikut dengan isi dari pendahuluan

Penelitian atau implementasi ini dilatarbelakangi oleh pesatnya
penggunaan \emph{cloud computing} serta kebutuhan untuk meningkatkan
utilisasi penggunaan komputer di laboratorium Departemen Teknik Informatika
Institut Teknologi Sepuluh Nopember.

\section{Latar Belakang}
\label{sec:latarbelakang}

Pesatnya perkembangan teknologi dalam bidang komputer
seperti komputasi awan memudahkan penggunaan sumber daya
komputasi secara dinamis. Teknologi komputasi awan memungkinkan
pengguna untuk menggunakan sumber daya komputasi sesuai dengan spesifikasi
yang dibutuhkan tanpa perlu memiliki sumber daya komputasi tersebut secara
fisik. Salah satu bentuk implementasi komputasi awan adalah
Google Cloud Platform (GCP).

GCP menyediakan banyak layanan komputasi awan seperti Compute Engine yang merupakan
layanan \emph{virtual machine} yang berjalan di mesin Google dan Google Kubernetes Engine (GKE)
yang merupakan layanan Kubernetes. Semua layanan tersebut merupakan bentuk implementasi
komputasi awan karena pengguna tidak perlu memiliki sumber daya tersebut secara
fisik namun menggunakan sumber daya komputasi yang ditawarkan oleh penyedia jasa seperti GCP.

Pada tugas akhir ini, akan dikembangkan sebuah sistem \emph{provisioning} untuk virtual machine
secara dinamis yang terletak di komputer Teknik Informatika ITS. \emph{Virtual machines}
yang dibuat akan secara otomatis tergabung dalam sebuah klaster Kubernetes yang nantinya
dapat digunakan oleh pengguna. 

\section{Permasalahan}
\label{sec:permasalahan}

Dari latar belakang tersebut maka dapat dijabarkan permasalahan sebagai berikut:

\begin{enumerate}[nolistsep]

  \item Bagaimana cara mengimplementasikan \emph{multi-tenancy} pada proses \emph{provisioning}
    klaster Kubernetes?

  \item Bagaimana cara untuk membuat \emph{virtual machine} yang tergabung
    dalam sebuah \emph{cluster} secara dinamis?

\end{enumerate}

\section{Tujuan}
\label{sec:Tujuan}

Tujuan dari penelitian atau implementasi ini adalah sebagai berikut:

\begin{enumerate}[nolistsep]

  \item Membuat sistem \emph{provisioning} yang dapat membuat \emph{virtual machine}
    secara dinamis sesuai dengan kriteria \emph{resource} permintaan pengguna.

\end{enumerate}

\section{Batasan Masalah}
\label{sec:batasanmasalah}

Batasan-batasan dari penelitian atau implementasi adalah sebagai berikut:

\begin{enumerate}[nolistsep]

  \item \emph{Provisioning} terjadi di lingkungan Teknik Informatika Institut
    Teknologi Sepuluh Nopember

\end{enumerate}

\section{Sistematika Penulisan}
\label{sec:sistematikapenulisan}

Laporan penelitian tugas akhir ini terbagi menjadi beberapa bagian seperti
berikut:

\begin{enumerate}[nolistsep]

  \item \textbf{BAB I Pendahuluan}

        Bab ini berisi pengantar sekaligus gambaran tugas akhir
        secara garis besar. Bab ini terdiri dari lima subbab, yaitu subbab
        latar belakang, subbab permasalahan, subbab tujuan, subbab batasan
        masalah, dan subbab sistematika penulisan.

        \vspace{2ex}

  \item \textbf{BAB II Tinjauan Pustaka}

        Bab ini berisi tinjauan pustaka dari teknologi-teknologi
        yang telah diciptakan sebelumnya untuk membantu dalam proses
        implementasi.

        \vspace{2ex}

  \item \textbf{BAB III Desain dan Implementasi Sistem}

        Bab ini berisi penjelasan desain dan implementasi dari sistem.

        \vspace{2ex}

  \item \textbf{BAB IV Pengujian dan Analisa}

        Bab ini berisi pengujian dan analisa dari sistem yang telah dibuat.
        Pengujian dan analisa dilakukan untuk mengetahui apakah sistem yang
        telah dibuat ... .

        \vspace{2ex}

  \item \textbf{BAB V Penutup}

        Bab ini berisi

\end{enumerate}
