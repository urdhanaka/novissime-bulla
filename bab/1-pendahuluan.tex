\chapter{PENDAHULUAN}
\label{chap:pendahuluan}

\section{Latar Belakang}
\label{sec:latarbelakang}

Pesatnya perkembangan teknologi dalam bidang komputer
seperti komputasi awan memudahkan penggunaan sumber daya
komputasi secara dinamis. Teknologi komputasi awan memungkinkan
pengguna untuk menggunakan sumber daya komputasi sesuai dengan spesifikasi
yang dibutuhkan tanpa perlu memiliki sumber daya komputasi tersebut secara
fisik. Salah satu bentuk implementasi komputasi awan adalah
Google Cloud Platform (GCP).

GCP menyediakan banyak layanan komputasi awan seperti Compute Engine yang merupakan
layanan \emph{virtual machine} yang berjalan di mesin Google dan Google Kubernetes Engine (GKE)
yang merupakan layanan Kubernetes. Semua layanan tersebut merupakan bentuk implementasi
komputasi awan karena pengguna tidak perlu memiliki sumber daya tersebut secara
fisik namun menggunakan sumber daya komputasi yang ditawarkan oleh penyedia jasa seperti GCP.

Teknologi komputasi awan sangat bermanfaat pada banyak studi kasus. Salah satunya adalah pengguna
dapat memiliki sumber daya komputasi tambahan kapan saja. Ketika pengguna ingin melakukan komputasi
yang harus dilakukan oleh sumber daya komputasi yang besar dan pengguna tersebut tidak memiliki
sumber daya komputasi tersebut secara fisik, pengguna tersebut dapat menyewa sumber daya komputasi
tambahan dari penyedia jasa sumber daya komputasi. Setelah proses komputasi tersebut selesai, pengguna
dapat mengembalikan sumber daya komputasi tersebut.

Di Teknik Informatika ITS, terdapat banyak komputer di lab yang bisa digunakan oleh mahasiswa saat
pembelajaran berlangsung. Komputer-komputer tersebut biasa digunakan pada saat terjadi proses pembelajaran
namun mati saat tidak terjadi proses pembelajaran. Untuk meningkatkan utilisasi dari komputer-komputer tersebut,
sistem \emph{provisioning} akan dikembangkan yang nantinya akan menggunakan komputer-komputer tersebut sebagai
\emph{worker} untuk klaster Kubernetes. Dengan begitu, komputer-komputer tersebut akan terutilisasi
meskipun kegiatan pembelajaran sedang tidak berlangsung. Selain itu, klaster yang dibuat
dapat digunakan sebagai tambahan sumber daya komputasi dalam melakukan komputasi yang berat.

Pada tugas akhir ini, akan dikembangkan sebuah sistem \emph{provisioning} untuk virtual machine
secara dinamis yang terletak di komputer Teknik Informatika ITS. \emph{Virtual machines} atau VM
yang dibuat akan secara otomatis tergabung dalam sebuah klaster Kubernetes yang nantinya
dapat digunakan oleh pengguna.

\section{Rumusan Permasalahan}
\label{sec:rumusanpermasalahan}

Dari latar belakang tersebut maka dapat dijabarkan permasalahan sebagai berikut:

\begin{enumerate}[nolistsep]

  \item Bagaimana cara merancang sistem \emph{provisioning} VM yang dapat diimplementasikan
    di lingkungan Teknik Informatika ITS?

  \item Bagaimana cara mengimplementasikan \emph{multi-tenancy} pada proses \emph{provisioning}
    klaster Kubernetes?

  \item Bagaimana cara untuk membuat \emph{virtual machine} yang tergabung
    dalam sebuah klaster secara dinamis?

\end{enumerate}

\section{Batasan Masalah}
\label{sec:batasanmasalah}

Batasan-batasan dari penelitian atau implementasi adalah sebagai berikut:

\begin{enumerate}[nolistsep]

  \item \emph{Provisioning} dilakukan di lingkungan Teknik Informatika Institut
    Teknologi Sepuluh Nopember

  \item Hasil implementasi berupa aplikasi \emph{provisioning} yang terdapat di setiap
    komputer yang akan digunakan dan aplikasi web sebagai \emph{dashboard}

  \item Kubernetes yang digunakan adalah K3s sebagai salah satu distribusi
    Kubernetes dengan versi v1.33.1+k3s1

  \item Aplikasi web berupa \emph{dashboard} dan aplikasi \emph{provisioning} dibuat
    menggunakan bahasa Golang versi 1.24.3

  \item Aplikasi \emph{provisioning} menggunakan libvirt dan libvirtd dengan versi 11.0.0

  \item Evaluasi dari sistem yang dibuat adalah 

\end{enumerate}

\section{Tujuan}
\label{sec:Tujuan}

Tujuan dari implementasi ini adalah untuk mengembangkan sistem \emph{provisioning}
yang dapat membuat VM dan secara otomatis tergabung ke dalam sebuah klaster Kubernetes secara dinamis sesuai dengan kriteria
\emph{resource} permintaan pengguna.

\section{Manfaat}
\label{sec:sistematikapenulisan}

Manfaat dari implementasi ini adalah untuk menyediakan sebuah sistem penyewaan
sumber daya komputasi yang berada di Teknik Informatika Institut Teknologi Sepuluh
Nopember. Diharapkan sistem penyewaan ini dapat digunakan sebagai sumber daya komputasi
tambahan yang dapat digunakan untuk proses komputasi yang berat.
