\chapter{KESIMPULAN DAN SARAN}
\label{chap:penutup}

Pada bab ini akan dipaparkan kesimpulan dari hasil implementasi
yang sudah dilakukan. Selain itu, saran mengenai hal yang dapat
dilakukan untuk mengembangkan implementasi ini juga akan dipaparkan.

\section{Kesimpulan}
\label{sec:kesimpulan}

Berdasarkan hasil implementasi yang dilakukan, berikut adalah
kesimpulan yang dapat diambil berdasarkan permasalahan yang diangkat
pada tugas akhir ini:

\begin{enumerate}[nolistsep]

  \item Pembuatan sistem \emph{provisioning} untuk menerapkan \emph{multi-tenancy}
    dalam klaster Kubernetes dapat dilakukan dengan cara mengaplikasikan \emph{tools}
    untuk membuat \emph{virtual machine} seperti Libvirt, Cloud-init, dan \emph{cloud images}.

  \item Penerapan \emph{multi-tenancy} pada implementasi ini adalah
    komputer fisik yang digunakan sebagai tempat \emph{provisioning} dapat
    membuat lebih dari satu \emph{virtual machine} yang tergabung dalam
    klaster yang berbeda. Implementasi tersebut menyebabkan komputer
    fisik tersebut melayani lebih dari satu \emph{user} sehingga
    komputer fisik tersebut memenuhi kriteria \emph{multi-tenancy}.

  \item Pembuatan \emph{virtual machine} yang secara otomatis tergabung
    dalam sebuah klaster Kubernetes dapat dicapai melalui beberapa \emph{tools}
    seperti K3s, Libvirt dan Cloud-init. K3s digunakan sebagai distribusi
    Kubernetes yang berukuran lebih kecil untuk membuat klaster Kubernetes,
    Libvirt digunakan untuk berinteraksi dengan hypervisor agar dapat membuat 
    \emph{virtual machine} dan Cloud-init digunakan untuk mengotomatisasi proses
    pembuatan dan penggabungan klaster Kubernetes.

\end{enumerate}

\section{Saran}
\label{chap:saran}

Untuk pengembangan lebih lanjut pada sistem \emph{provisioning}
untuk membuat klaster Kubernetes adalah sebagai berikut:

\begin{enumerate}[nolistsep]

  \item Meningkatkan kualitas dari antarmuka pengguna dan pengalaman
    pengguna pada situs web \emph{dashboard} sebagai antarmuka
    yang digunakan oleh \emph{user} untuk membuat klaster.

  \item Meningkatkan keamanan dengan cara mengisolasi \emph{virtual machine}
    agar tidak dapat diakses oleh \emph{virtual machine} lain yang tidak
    dalam satu klaster.

  \item Mengutilisasi GPU dalam pembuatan \emph{virtual machine} sehingga
    GPU dapat digunakan sebagai salah satu sumber daya pada \emph{virtual cluster}.

\end{enumerate}
