\chapter{KESIMPULAN DAN SARAN}
\label{chap:penutup}

Pada bab ini akan dipaparkan kesimpulan dari hasil implementasi
yang sudah dilakukan. Selain itu, saran mengenai hal yang dapat
dilakukan untuk mengembangkan implementasi ini juga akan dipaparkan.

\section{Kesimpulan}
\label{sec:kesimpulan}

Berdasarkan hasil implementasi yang dilakukan, berikut adalah
kesimpulan yang dapat diambil berdasarkan permasalahan yang diangkat
pada tugas akhir ini:

\begin{enumerate}[nolistsep]

  \item Pembuatan \emph{virtual cluster} Kubernetes dengan melakukan \emph{resource allocations}
    dapat dilakukan dengan cara membuat VM yang memiliki sumber daya sesuai dengan
    sumber daya yang sudah ditentukan. Setelah semua VM tersebut terbentuk, maka \emph{virtual cluster}
    juga akan terbentuk dengan \emph{control plane} dan \emph{node} sesuai dengan VM yang telah dibuat
    sebelumnya.

  \item \emph{Multi-tenancy} dapat diimplementasikan pada \emph{virtual cluster}
    dengan cara menggunakan Linux \emph{bridge}, sehingga setiap VM yang diciptakan
    untuk \emph{virtual cluster} memiliki alamat IP dengan jangkauan yang sama dengan
    \emph{host}. Hal tersebut ditujukan agar VM yang berada di komputer yang berbeda dapat
    berkomunikasi. Setelah itu, permintaan pembuatan VM untuk \emph{virtual cluster}
    tidak hanya diberikan ke satu \emph{worker node} saja, sehingga dalam satu \emph{virtual cluster},
    dua VM atau lebih bisa saja berada di komputer yang berbeda. Oleh karena itu, dalam satu \emph{worker node},
    bisa saja terdiri dari VM yang dibuat oleh \emph{user} yang berbeda yang juga tergabung
    di \emph{virtual cluster} yang berbeda.

  \item Pembuatan \emph{virtual machine} yang secara otomatis tergabung
    dalam sebuah klaster Kubernetes dapat dicapai melalui beberapa \emph{tools}
    seperti K3s, Libvirt dan Cloud-init. K3s digunakan sebagai distribusi
    Kubernetes yang berukuran lebih kecil untuk membuat klaster Kubernetes,
    Libvirt digunakan untuk berinteraksi dengan hypervisor agar dapat membuat 
    \emph{virtual machine} dan Cloud-init digunakan untuk mengotomatisasi proses
    pembuatan dan penggabungan klaster Kubernetes.

\end{enumerate}

\section{Saran}
\label{chap:saran}

Untuk pengembangan lebih lanjut pada sistem \emph{provisioning}
untuk membuat klaster Kubernetes adalah sebagai berikut:

\begin{enumerate}[nolistsep]

  \item Meningkatkan kualitas dari antarmuka pengguna dan pengalaman
    pengguna pada situs web \emph{dashboard} sebagai antarmuka
    yang digunakan oleh \emph{user} untuk membuat klaster.

  \item Meningkatkan keamanan dengan cara mengisolasi \emph{virtual machine}
    agar tidak dapat diakses oleh \emph{virtual machine} lain yang tidak
    dalam satu klaster.

  \item Mengutilisasi GPU dalam pembuatan \emph{virtual machine} sehingga
    GPU dapat digunakan sebagai salah satu sumber daya pada \emph{virtual cluster}.

\end{enumerate}
