\chapter{HASIL DAN PEMBAHASAN}
\label{chap:hasil-pembahasan}

Pada bab ini akan dijelaskan mengenai hasil implementasi
dan pengujian dari sistem \emph{provisioning} yang telah
dibuat sebelumnya pada bab 3. Pengujian yang dilakukan
menggunakan laptop penulis dan komputer dari laboratorium Rekayasa Perangkat
Lunak Teknik Informatika ITS. Spesifikasi dari laptop dan komputer
yang digunakan dapat dilihat pada tabel \ref{tb:spesifikasi-laptop-pengujian} dan
\ref{tb:spesifikasi-komputer-pengujian}

\begin{longtable}{|c|c|}
  \caption{Spesifikasi Laptop untuk Pengujian}
  \label{tb:spesifikasi-laptop-pengujian} \\
  \hline
  OS     & Fedora Linux 42.20250614.0 (Kinoite) x86\_64 \\
  \hline
  Kernel & Linux 6.14.9-300.fc42.x86\_64                \\
  \hline
  CPU    & Intel(R) Core(TM) i7-10750H (12) @ 5.00 GHz       \\
  \hline
  Integrated GPU   & Intel UHD Graphics @ 1.15 GHz [Integrated]       \\
  \hline
  Discrete GPU    & NVIDIA GeForce GTX 1650 Ti Mobile [Discrete]       \\
  \hline
  RAM    & 15848MiB       \\
  \hline
\end{longtable}

\begin{longtable}{|c|c|}
  \caption{Spesifikasi Komputer untuk Pengujian}
  \label{tb:spesifikasi-komputer-pengujian} \\
  \hline
  OS     & Ubuntu 24.04.2 LTS x86\_64 \\
  \hline
  Kernel & 6.11.0-26-generic          \\
  \hline
  CPU    & 12th Gen Intel i7-12700 (20) @ 4.800GHz       \\
  \hline
  GPU    & Intel AlderLake-S GT1       \\
  \hline
  RAM    & 31834MiB       \\
  \hline
\end{longtable}

\section{Hasil Implementasi}
\label{sec:hasil-implementasi}

\subsection{Implementasi Linux Bridge}
\label{subsec:implementasi-linux-bridge}

Linux \emph{bridge} yang telah dijelaskan sebelumnya dibuat
agar \emph{virtual machine} memiliki jangkauan alamat IP yang sama
dengan jangkauan alamat IP dari komputer \emph{host}. Jangkauan IP yang
sama dengan komputer \emph{host} diperlukan agar \emph{virtual machine}
yang berada di komputer A dapat berkomunikasi dengan \emph{virtual machine}
yang berada di komputer B.

\begin{lstlisting}[
  style=clistyle,
  caption={Informasi \emph{Network Interface} pada Komputer \emph{Host}},
  label={cli:host-network-interface}
]
...
2: enp4s0: <BROADCAST,MULTICAST,UP,LOWER_UP> mtu 1500 qdisc mq master k3s-br0 state UP group default qlen 1000
    link/ether c8:7f:54:6c:47:ff brd ff:ff:ff:ff:ff:ff
...
8: k3s-br0: <BROADCAST,MULTICAST,UP,LOWER_UP> mtu 1500 qdisc noqueue state UP group default qlen 1000
    link/ether 92:e9:0d:63:4e:32 brd ff:ff:ff:ff:ff:ff
    inet 10.21.73.107/24 brd 10.21.73.255 scope global dynamic noprefixroute k3s-br0
       valid_lft 94877sec preferred_lft 94877sec
    inet6 fe80::1d88:4787:24c3:2a99/64 scope link noprefixroute
       valid_lft forever preferred_lft forever
\end{lstlisting}

Pada kode sumber \ref{cli:host-network-interface}, komputer \emph{host} memiliki
\emph{network interface} untuk ethernet bernama enp4s0. Linux \emph{bridge} yang
dibuat untuk implementasi tugas akhir ini adalah k3s-br0 dan enp4s0 menjadi
\emph{slave} dari Linux \emph{bridge} k3s-br0. Alamat IP dan jangkauan IP dari
komputer \emph{host} tersebut adalah 10.21.73.17 dengan \emph{subnet mask} /24
yang berarti dalam \emph{local area network} tersebut terdapat 255 alamat
IP yang dapat dibagikan oleh DHCP server.

Agar \emph{virtual machine} memiliki jangkauan alamat IP yang sama dengan 
jangkauan alamat IP address yang didapat oleh komputer \emph{host}, \emph{virtual machine}
dapat menggunakan Linux \emph{bridge} yang sudah dibuat sebagai sumber jaringan
yang dipakai. Konfigurasi tersebut dapat dilakukan melalui konfigurasi berbentuk
xml seperti yang dapat dilihat pada kode sumber \ref{code:vm-interface-configuration}.
Alamat IP yang didapat pada \emph{virtual machine} dapat dilihat pada kode
sumber \ref{cli:vm-network-interface}.

\begin{lstlisting}[
  language=xml,
  style=codestyle,
  caption={Konfigurasi \emph{Interface} Jaringan pada \emph{Virtual Machine}},
  label={code:vm-interface-configuration}
]
...
<interface type='bridge'>
  <mac address='52:54:00:cf:ee:9e'/>
  <source bridge='k3s-br0'/>
  <model type='virtio'/>
  <alias name='net0'/>
  <address type='pci' domain='0x0000' bus='0x01' slot='0x00' function='0x0'/>
</interface>
...
\end{lstlisting}

\begin{lstlisting}[
  style=clistyle,
  caption={Informasi \emph{Network Interface} pada \emph{Virtual Machine}},
  label={cli:vm-network-interface}
]
...
2: enp1s0: <BROADCAST,MULTICAST,UP,LOWER_UP> mtu 1500 qdisc fq_codel state UP group default qlen 1000
    link/ether 52:54:00:cf:ee:9e brd ff:ff:ff:ff:ff:ff
    inet 10.21.73.134/24 brd 10.21.73.255 scope global dynamic noprefixroute enp1s0
       valid_lft 105859sec preferred_lft 105859sec
    inet6 fe80::5054:ff:fecf:ee9e/64 scope link proto kernel_ll
       valid_lft forever preferred_lft forever
...
\end{lstlisting}

Pada kode sumber \ref{cli:vm-network-interface}, alamat IP yang didapat oleh
\emph{virtual machine} adalah 10.21.73.134 dan memiliki \emph{subnet mask} /24.
Alamat IP dan \emph{subnet mask} yang didapat oleh \emph{virtual machine} sama
dengan alamat IP dan \emph{subnet mask} yang didapat oleh komputer \emph{host}.
Informasi lanjutan mengenai Linux \emph{bridge} dan \emph{network interfaces}
yang tersambung dapat dilihat pada kode sumber \ref{cli:bridge-interfaces}.

\begin{lstlisting}[
  style=clistyle,
  caption={Linux \emph{Bridge} dan \emph{Network Interfaces} yang Tersambung},
  label={cli:bridge-interfaces}
]
bridge name     bridge id               STP enabled     interfaces
k3s-br0         8000.92e90d634e32       yes             enp4s0
                                                        vnet7
\end{lstlisting}

Pada kode sumber \ref{cli:bridge-interfaces}, \emph{network interfaces} yang
berada di bawah Linux \emph{bridge} k3s-br0 adalah \emph{network interface}
enp4s0 dan vnet7. \emph{Network interface} enp4s0 merupakan \emph{network interface}
untuk ethernet sedangkan vnet7 merupakan \emph{virtual network} yang dibuat
dan tersambung dengan \emph{network interface} dari \emph{virtual machine}.
Gambaran mengenai hal tersebut dapat dilihat pada \ref{cli:vnet-big}.

\begin{lstlisting}[
  style=clistyle,
  caption={Hubungan Antar \emph{Devices}},
  label={cli:vnet-big}
]
VM's ethernet interface <-> vnet <-> Linux bridge <-> physical NIC
\end{lstlisting}

\subsection{Implementasi Cloud-init}
\label{subsec:implementasi-cloud-init}

Setiap \emph{virtual machine} yang dibuat memiliki sebuah \emph{file} berekstensi iso khusus
yang berisikan \emph{file} konfigurasi Cloud-init. \emph{File} tersebut kemudian akan digunakan
saat mendefinisikan konfigurasi xml dari \emph{virtual machine}. Contoh definisi konfigurasi
xml tersebut dapat dilihat pada kode sumber \ref{lst:xml-cloud-init}.

\lstinputlisting[
  language=go,
  style=codestyle,
  emphstyle=\color{black}\bfseries\underbar,
  emph={seedFile},
  caption={Konfigurasi xml dengan \emph{File} Cloud-init},
  label={lst:xml-cloud-init}
]{program/xml-cloud-init-iso.go}

Cloud-init yang berjalan pada \emph{virtual machine} menghasilkan \emph{file}
\emph{log} dari proses konfigurasi Cloud-init. \emph{File} tersebut dapat digunakan
untuk proses \emph{debug} konfigurasi yang digunakan. Cloud-init menghasilkan
dua \emph{file logging}, yaitu \emph{log} dari proses Cloud-init itu sendiri
dan \emph{log} dari \emph{command} yang dijalankan oleh Cloud-init. Hasil
dari dua \emph{file} tersebut dapat dilihat pada gambar \ref{fig:log-cloud-init}
dan gambar \ref{fig:log-output-cloud-init}.

\begin{figure}[H]
  \centering
  \fbox{\includegraphics[scale=0.3]{gambar/cloud-init.png}}
  \caption{\emph{Log} Cloud-init}
  \label{fig:log-cloud-init}
\end{figure}

\begin{figure}[H]
  \centering
  \fbox{\includegraphics[scale=0.3]{gambar/cloud-init-output.png}}
  \caption{\emph{Log Command} yang Dijalankan Cloud-init}
  \label{fig:log-output-cloud-init}
\end{figure}
\subsection{Pengujian Pembuatan \emph{Virtual Cluster} Lingkungan Lokal}

\label{subsec:pengujian-pembuatan-vc}

Pengujian proses \emph{provisioning} pada komputer lokal bertujuan untuk
menguji apakah sistem yang telah dibangun dapat membuat \emph{virtual cluster}
yang sesuai dengan kriteria \emph{user}. Pengujian pada lingkungan lokal ini
tidak memerlukan Linux \emph{bridge} karena jaringan internet yang digunakan
oleh \emph{virtual machine} akan menggunakan \emph{Network Address Translation} (NAT)
dari \emph{host}, sehingga setiap \emph{virtual machine} dapat berkomunikasi satu sama
lain tanpa menggunakan Linux \emph{bridge}.

Skenario pengujian yang akan dilakukan pada lingkungan lokal adalah
membuat klaster Kubernetes dengan satu \emph{virtual machine} dan dua \emph{virtual machine}.
Pada klaster Kubernetes yang terdiri satu \emph{virtual machine}, \emph{virtual machine}
tersebut bertugas sebagai \emph{control plane}. Sedangkan pada klaster Kubernetes yang terdiri
dari dua \emph{virtual machine}, satu dari \emph{virtual machine} tersebut bertugas sebagai
\emph{control plane} dan sisanya sebagai \emph{worker node}.

Setelah pembuatan klaster selesai, \emph{dashboard} akan menampilkan
token untuk dapat mengakses \emph{dashboard} Kubernetes dari klaster
yang dibuat. Selain itu, tombol untuk mengakses klaster juga akan ditampilkan
dan dapat ditekan oleh \emph{user} untuk menuju situs web \emph{dashboard}
klaster Kubernetes.

\begin{figure}[H]
  \centering
  \fbox{\includegraphics[scale=0.3]{gambar/website-create-process.png}}
  \caption{Proses Pembuatan Klaster}
  \label{fig:proses-pembuatan-klaster}
\end{figure}

\begin{figure}[H]
  \centering
  \fbox{\includegraphics[scale=0.3]{gambar/website-create-process-done-local.png}}
  \caption{Proses Pembuatan Klaster Selesai}
  \label{fig:proses-pembuatan-klaster-selesai}
\end{figure}

\begin{figure}[H]
  \centering
  \includegraphics[scale=0.3]{gambar/worker-create-cluster-process-local.png}
  \caption{\emph{Log} pada Komputer \emph{Worker}}
  \label{fig:worker-create-cluster-process-local}
\end{figure}

\begin{figure}[H]
  \centering
  \includegraphics[scale=0.3]{gambar/kubernetes-dashboard-access-local-with-nodes.png}
  \caption{Daftar \emph{Nodes} pada Klaster dengan Satu \emph{Virtual Machine}}
  \label{fig:daftar-nodes-pada-dashboard-kubernetes}
\end{figure}

Berdasarkan gambar-gambar di atas, sistem \emph{provisioning} dapat membuat
\emph{virtual machine} yang secara otomatis tergabung dalam sebuah klaster Kubernetes.
Selain itu, \emph{control plane} juga menyediakan \emph{dashboard} Kubernetes yang dapat
digunakan oleh pengguna untuk berinteraksi dengan klaster Kubernetes tersebut.

\subsection{Pengujian Pembuatan \emph{Virtual Cluster} Lingkungan \emph{Production}}
\label{subsec:pengujian-pembuatan-vc-prod}

Pada lingkungan \emph{production}, \emph{virtual machine} yang tergabung dalam
satu klaster tidak selalu berada dalam satu komputer fisik \emph{worker} yang sama.
Pengujian dilakukan dengan cara membuat klaster berisi dua atau lebih \emph{virtual machine}
yang terdiri dari satu \emph{control plane} dan sisanya sebagai \emph{worker node}.
Semua \emph{Virtual machine} tersebut tidak selalu berada di komputer \emph{worker}
yang sama.

Pada subbab pengujian ini, akan dibuat sebuah klaster yang terdiri dari dua
\emph{virtual machine} yang berada di dua komputer fisik yang berbeda. Pada gambar
\ref{fig:nodes-2-komputer-berbeda}, klaster tersebut memiliki \emph{nodes} yang
bernama eovugekt dan gxlshrqm. Komputer fisik dari dua \emph{virtual machine}
tersebut dapat dilihat pada gambar \ref{fig:vm-komputer-fisik-1}.

\begin{figure}[H]
  \centering
  \includegraphics[scale=0.3]{gambar/two-nodes-difference-computer-dashboard.png}
  \caption{Daftar \emph{Nodes} pada Klaster 1}
  \label{fig:nodes-2-komputer-berbeda-1}
\end{figure}

\begin{figure}[H]
  \centering
  \includegraphics[scale=0.3]{gambar/ssh-nodes-list-1.png}
  \caption{Daftar \emph{Virtual Machine} Setelah Pembuatan Klaster 1}
  \label{fig:vm-komputer-fisik-1}
\end{figure}

Gambar \ref{fig:vm-komputer-fisik-1} menunjukkan bahwa \emph{nodes} gxlshrqm berada
pada komputer fisik rpl-1 dan \emph{nodes} eovugekt berada pada
komputer fisik rpl-02. Untuk pengujian \emph{multi-tenancy}, akan
dibuat klaster Kubernetes lagi dari dua komputer fisik tersebut.

\begin{figure}[H]
  \centering
  \includegraphics[scale=0.3]{gambar/two-nodes-difference-computer-dashboard-2.png}
  \caption{Daftar \emph{Nodes} pada Klaster 2}
  \label{fig:nodes-2-komputer-berbeda-2}
\end{figure}

\begin{figure}[H]
  \centering
  \includegraphics[scale=0.3]{gambar/ssh-nodes-list-2.png}
  \caption{Daftar \emph{Virtual Machine} Setelah Pembuatan Klaster 2}
  \label{fig:vm-komputer-fisik-2}
\end{figure}

Pada gambar \ref{fig:nodes-2-komputer-berbeda-2}, klaster yang baru
memiliki \emph{nodes} bernama qxwuyusc dan vsbftqms. Gambar \ref{fig:vm-komputer-fisik-2}
menunjukkan bahwa \emph{node} qxwuyusc berada pada komputer fisik rpl-02 yang juga
merupakan komputer fisik dari \emph{node} eovugekt. Dari hal tersebut, dapat
dilihat bahwa komputer fisik rpl-02 digunakan oleh dua klaster dan dua pengguna
yang berbeda.

\subsection{\emph{Error Handling}}
\label{subsec:error-handling}

Proses \emph{provisioning virtual machine} untuk pembuatan
klaster Kubernetes tidak selalu berjalan baik. Terdapat banyak
faktor yang dapat membuat proses \emph{provisioning} gagal. Untuk
mengatasi hal tersebut, proses \emph{provisioning} pada implementasi
tugas akhir ini memiliki sifat atomik, yaitu jika terjadi \emph{error}
saat membuat \emph{virtual machine} dan \emph{virtual machine} sudah terbuat,
\emph{virtual machine} tersebut akan dihapus.

\lstinputlisting[
  language=go,
  style=codestyle,
  caption={Kode Sumber Penghapusan \emph{Virtual Machine}},
  label={lst:vm-delete-function}
]{program/delete-vm.go}

\lstinputlisting[
  language=go,
  style=codestyle,
  emphstyle=\color{black}\bfseries\underbar,
  emph={deleteInstance},
  caption={Contoh Penggunaan \emph{Error Handling}},
  label={lst:error-handling-example}
]{program/error-handling.go}

Pada kode sumber \ref{lst:vm-delete-function}, fungsi \lstinline{deleteInstance}
akan mematikan \emph{virtual machine} dan menghapus \emph{file} yang berkaitan
dengan \emph{virtual machine} tersebut seperti \emph{cloud image} yang dipakai.
Bentuk implementasi dari \emph{error handling} tersebut dapat dilihat pada kode
sumber \ref{lst:error-handling-example}. Pada kode sumber tersebut, jika terjadi
\emph{error} pada saat membuat \emph{virtual machine}, maka \emph{virtual machine}
tersebut akan dihapus beserta semua \emph{file} yang berkaitan.

% Contoh pembuatan tabel
% \begin{longtable}{|c|c|c|}
%   \caption{Hasil Pengukuran Energi dan Kecepatan}
%   \label{tb:EnergiKecepatan}                                   \\
%   \hline
%   \rowcolor[HTML]{C0C0C0}
%   \textbf{Energi} & \textbf{Jarak Tempuh} & \textbf{Kecepatan} \\
%   \hline
%   10 J            & 1000 M                & 200 M/s            \\
%   20 J            & 2000 M                & 400 M/s            \\
%   30 J            & 4000 M                & 800 M/s            \\
%   40 J            & 8000 M                & 1600 M/s           \\
%   \hline
% \end{longtable}
