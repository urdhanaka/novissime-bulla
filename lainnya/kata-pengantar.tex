\chapter*{KATA PENGANTAR}

\addcontentsline{toc}{chapter}{KATA PENGANTAR}

\vspace{2ex}

% Ubah paragraf-paragraf berikut dengan isi dari kata pengantar

Puji dan syukur penulis panjatkan kepada Tuhan Yang Maha Esa karena atas rahmat
dan karunia-Nya, penulis dapat menyelesaikan laporan tugas akhir yang berjudul
"IMPLEMENTASI \emph{MULTI-TENANCY} UNTUK PROVISIONING KLASTER KUBERNETES" dengan
baik. Tugas akhir ini disusun sebagai persyaratan untuk memperoleh gelar sarjana
pada program studi Teknik Informatika Fakultas Teknologi Elektro dan Informatika
Cerdas Institut Teknologi Sepuluh Nopember. Penulisan tugas akhir ini tidak lepas
dari dukungan dan bantuan dari berbagai pihak.
Oleh karena itu, penulis mengucapkan terima kasih kepada:

\begin{enumerate}[nolistsep]

  \item Bapak Royyana Muslim Ijtihadie, S.Kom., M.Kom., Ph.D. selaku dosen pembimbing
    pertama penulis dan Bapak Ary Mazharuddin Shiddiqi, S.Kom., M.Comp.Sc., Ph.D. selaku dosen
    pembimbing kedua penulis yang telah membantu penulis dalam menyelesaikan Tugas Akhir.

  \item Bapak 

  \item Bapak Agung Budi Cahyono dan Ibu Isti Purwaningsih beserta dua saudara penulis
    yang telah memberikan dukungan kepada penulis.

  % \item Teman-teman "\emph{Backstreet Boys}"
    
  \item Alwan Raihan, Bhatara Arundaya, Orisvio Revanda, S.T., Rayhan Rizky, dan Tigo Yoga
    selaku teman-teman "Ahmad Dahlan" yang telah memberikan semangat dan dukungan kepada
    penulis selama pengerjaan.

  \item Adam Daffa, Hanif Setyadi, Muhammad Rafi, Orisvio Revanda, S.T., dan Tio
    Widayat, S.H., selaku teman "Risin Inti Diing". Terima kasih penulis ucapkan
    sebesar-besarnya.

\end{enumerate}

Laporan tugas akhir ini jauh dari kata sempurna, oleh karena itu penulis
sangat terbuka terhadap kritik dan saran yang membangun. Akhir kata, semoga
laporan tugas akhir ini dapat memberikan dampak dan manfaat yang baik kepada
seluruh pihak yang memerlukan.

\begin{flushright}
  \begin{tabular}[b]{c}
    \place{}, \MONTH{} \the\year{} \\
    \\
    \\
    \\
    \\
    \name{}
  \end{tabular}
\end{flushright}
