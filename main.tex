% Judul dokumen
\title{Buku Tugas Akhir ITS}
\author{Aptanagi, Urdhanaka}

% Pengaturan ukuran teks dan bentuk halaman dua sisi
\documentclass[12pt,twoside]{report}
\setcounter{secnumdepth}{3}
\setcounter{tocdepth}{2}

% Pengaturan ukuran halaman dan margin
\usepackage[a4paper,top=30mm,left=30mm,right=20mm,bottom=25mm]{geometry}

% Pengaturan ukuran spasi
\usepackage[singlespacing]{setspace}

% Pengaturan detail pada file PDF
\usepackage[pdfauthor={\@author},bookmarksnumbered,pdfborder={0 0 0}]{hyperref}

% Pengaturan jenis karakter
\usepackage[T1]{fontenc}
\usepackage[utf8]{inputenc}

% Pengaturan pewarnaan
\usepackage[table,xcdraw]{xcolor}

% Pengaturan kutipan artikel
\usepackage[style=apa, backend=biber]{biblatex}

\usepackage[titles]{tocloft}
\setlength{\cftsecindent}{1em}
\setlength{\cftsubsecindent}{2em}
\setlength{\cftsubsubsecindent}{3em}
\setlength{\cftbeforechapskip}{1.5ex}
\setlength{\cftbeforesecskip}{1ex}
\setlength{\cftbeforesubsecskip}{1ex}
\setlength{\cftbeforesubsubsecskip}{1ex}
\setlength{\cftbeforetoctitleskip}{0cm}
\setlength{\cftbeforeloftitleskip}{4ex}
\setlength{\cftafterloftitleskip}{0cm}
\setlength{\cftbeforelottitleskip}{0cm}
\setlength{\cftfigindent}{0pt}
\setlength{\cfttabindent}{0pt}
\renewcommand{\cfttoctitlefont}{\hfill\Large\bfseries}
\renewcommand{\cftaftertoctitle}{\hfill}
\renewcommand{\cftloftitlefont}{\hfill\Large\bfseries}
\renewcommand{\cftafterloftitle}{\hfill}
\renewcommand{\cftlottitlefont}{\hfill\Large\bfseries}
\renewcommand{\cftafterlottitle}{\hfill}
\renewcommand{\cftchappresnum}{BAB\ }
\setlength{\cftchapnumwidth}{3.2em}

% Package lainnya
\usepackage{changepage}
\usepackage{enumitem}
\usepackage{eso-pic}
\usepackage{txfonts} % Font times
\usepackage{etoolbox}
\usepackage{graphicx}
\usepackage{lipsum}
\usepackage{longtable}
\usepackage{tabularx}
\usepackage{wrapfig}
\usepackage{float}

% Definisi untuk "Hati ini sengaja dikosongkan"
\patchcmd{\cleardoublepage}{\hbox{}}{
  \thispagestyle{empty}
  \vspace*{\fill}
  \begin{center}\textit{[Halaman ini sengaja dikosongkan]}\end{center}
  \vfill}{}{}

% Pengaturan penomoran halaman
\usepackage{fancyhdr}
\fancyhf{}
\renewcommand{\headrulewidth}{0pt}
\pagestyle{fancy}
\fancyfoot[LE,RO]{\thepage}
\patchcmd{\chapter}{plain}{fancy}{}{}
\patchcmd{\chapter}{empty}{plain}{}{}

% Command untuk bulan
\newcommand{\MONTH}{%
  \ifcase\the\month
  \or Januari% 1
  \or Februari% 2
  \or Maret% 3
  \or April% 4
  \or Mei% 5
  \or Juni% 6
  \or Juli% 7
  \or Agustus% 8
  \or September% 9
  \or Oktober% 10
  \or November% 11
  \or Desember% 12
  \fi}
\newcommand{\ENGMONTH}{%
  \ifcase\the\month
  \or January% 1
  \or February% 2
  \or March% 3
  \or April% 4
  \or May% 5
  \or June% 6
  \or July% 7
  \or August% 8
  \or September% 9
  \or October% 10
  \or November% 11
  \or December% 12
  \fi}

% Pengaturan format judul bab
\usepackage{titlesec}
\titleformat{\chapter}[block]{\bfseries\Large}{BAB \centering\arabic{chapter}}{1ex}{\vspace{0ex}\centering}
\titleformat{\section}{\bfseries\large}{\MakeUppercase{\thesection}}{1ex}{\vspace{1ex}}
\titleformat{\subsection}{\bfseries\large}{\MakeUppercase{\thesubsection}}{1ex}{}
\titleformat{\subsubsection}{\bfseries\large}{\MakeUppercase{\thesubsubsection}}{1ex}{}
\titleformat{\subsubsubsection}{\bfseries\large}{\MakeUppercase{\thesubsubsubsection}}{1ex}{}
\titlespacing{\chapter}{0ex}{-5ex}{4ex}
\titlespacing{\section}{0ex}{1ex}{0ex}
\titlespacing{\subsection}{0ex}{0.5ex}{0ex}
\titlespacing{\subsubsection}{0ex}{0.5ex}{0ex}
\titlespacing{\subsubsubsection}{0ex}{0.5ex}{0ex}

\counterwithin{figure}{chapter}
\counterwithin{table}{chapter}

\usepackage{listings}
\definecolor{comment}{RGB}{0,128,0}
\definecolor{string}{RGB}{255,0,0}
\definecolor{keyword}{RGB}{0,0,255}
\renewcommand{\lstlistingname}{Kode Sumber}
\lstdefinestyle{codestyle}{
  commentstyle=\color{comment},
  stringstyle=\color{string},
  keywordstyle=\color{keyword},
  basicstyle=\footnotesize\ttfamily,
  numbers=left,
  numberstyle=\tiny,
  numbersep=5pt,
  frame=lines,
  breaklines=true,
  prebreak=\raisebox{0ex}[0ex][0ex]{\ensuremath{\hookleftarrow}},
  showstringspaces=false,
  upquote=true,
  tabsize=2,
  captionpos=b,
  % xleftmargin=5mm,
}
\lstdefinestyle{clistyle}{
  basicstyle=\footnotesize\ttfamily,
  numbers=none,
  frame=single,
  breaklines=true,
  prebreak=\raisebox{0ex}[0ex][0ex]{\ensuremath{\hookleftarrow}},
  captionpos=b,
}
\lstset{
  aboveskip=15pt,
  style=codestyle
}

% lstlistoflistings configuration
\makeatletter
\renewcommand{\l@lstlisting}[2]{
  \@dottedtocline{1}{0em}{2.3em}{#1}{#2}%
}
\makeatother

% Atur variabel berikut sesuai namanya

% nama
\newcommand{\name}{Urdhanaka Aptanagi}
\newcommand{\authorname}{Aptanagi, Urdhanaka}
\newcommand{\nickname}{Elon}
\newcommand{\advisor}{Royyana Muslim Ijtihadie, S.Kom., M.Kom., Ph.D.}
\newcommand{\coadvisor}{Ary Mazharuddin Shiddiqi, S.Kom., M.Comp.Sc., Ph.D.}
\newcommand{\examinerone}{Dr. Galileo Galilei, S.T., M.Sc}
\newcommand{\examinertwo}{Friedrich Nietzsche, S.T., M.Sc}
\newcommand{\examinerthree}{Alan Turing, ST., MT}
\newcommand{\headofdepartment}{Ary Mazharuddin Shiddiqi, S.Kom., M.Comp.Sc., Ph.D.}

  % Dosen Pembimbing&:& 1. Royyana Muslim Ijtihadie, S.Kom., M.Kom., Ph.D.\\

% identitas
\newcommand{\nrp}{5025211123}
\newcommand{\advisornip}{19770824 200304 1 001}
\newcommand{\coadvisornip}{19810620 200501 1 003}
\newcommand{\examineronenip}{18560710 194301 1 001}
\newcommand{\examinertwonip}{18560710 194301 1 001}
\newcommand{\examinerthreenip}{18560710 194301 1 001}
\newcommand{\headofdepartmentnip}{19810620 200501 1 003}

% judul
\newcommand{\tatitle}{IMPLEMENTASI \emph{MULTI-TENANCY} UNTUK PROVISIONING KLASTER KUBERNETES}
\newcommand{\engtatitle}{\emph{MULTI-TENANCY IMPLEMENTATION FOR KUBERNETES CLUSTER PROVISIONING}}

% tempat
\newcommand{\place}{Surabaya}

% jurusan
\newcommand{\studyprogram}{Teknik Informatika}
\newcommand{\engstudyprogram}{Informatics}

% fakultas
\newcommand{\faculty}{Teknologi Elektro dan Informatika Cerdas}
\newcommand{\engfaculty}{Intelligent Electrical and Informatics Technology}

% singkatan fakultas
\newcommand{\facultyshort}{FTEIC}
\newcommand{\engfacultyshort}{ELECTICS}

% departemen
\newcommand{\department}{Teknik Informatika}
\newcommand{\engdepartment}{Informatics Engineering}

% kode mata kuliah
\newcommand{\coursecode}{EF234801}

% Tambahkan format tanda hubung yang benar di sini
\hyphenation{
  ro-ket
  me-ngem-bang-kan
  per-hi-tu-ngan
  tek-no-lo-gi
  me-la-ku-kan
  ber-so-si-al-i-sa-si
  peng-hu-bung
  pro-ses
  No-pem-ber
  ber-na-ma
  di-kem-bang-kan
  klas-ter
  ge-ne-rik
  kon-ku-ren-si
  ku-be-con-trol-ler-ma-na-ger
  be-ri-kut
  per-be-da-an
  se-per-ti
  me-mu-dah-kan
  le-bih
  peng-gu-na-an
  Peng-gu-na-an
  Ble-ki-nge
  di-te-rus-kan
  Wor-ker
  Go-lang
  jang-kau-an
  ber-i-si-kan
  meng-gu-na-kan
  pe-nyim-pa-nan
  me-ngi-rim
}


% Menambahkan resource daftar pustaka
\addbibresource{pustaka/pustaka.bib}

% Isi keseluruhan dokumen
\begin{document}

% Sampul luar Bahasa Indonesia
\newcommand\covercontents{sampul/konten-id.tex}
\input{sampul/sampul-luar.tex}

% Atur ulang penomoran halaman
\setcounter{page}{1}

% Sampul dalam Bahasa Indonesia
\renewcommand\covercontents{sampul/konten-id-dalam.tex}
\input{sampul/sampul-luar-tipis.tex}
\cleardoublepage

% Sampul dalam Bahasa Inggris
\renewcommand\covercontents{sampul/konten-en.tex}
\input{sampul/sampul-luar-tipis.tex}
\cleardoublepage

% Label tabel dan gambar dalam bahasa indonesia
\renewcommand{\figurename}{Gambar}
\renewcommand{\tablename}{Tabel}

% Pengaturan ukuran indentasi paragraf
\setlength{\parindent}{2em}

% Pengaturan ukuran spasi paragraf
\setlength{\parskip}{1ex}

% Nomor halaman pembuka dimulai dari sini
\pagenumbering{roman}

% Lembar pengesahan dan pernyataan-keaslian
\chapter*{LEMBAR PENGESAHAN}

\addcontentsline{toc}{chapter}{LEMBAR PENGESAHAN}

\begin{center}
  \textbf{\tatitle{}}
\end{center}

\begingroup
% Pemilihan font ukuran small
\small

\begin{center}
  \textbf{TUGAS AKHIR}
  \\Diajukan untuk memenuhi salah satu syarat \\
  memperoleh gelar Sarjana Komputer pada \\
  Program Studi S-1 \studyprogram{} \\
  Departemen \department{} \\
  Fakultas \faculty{} \\
  Institut Teknologi Sepuluh Nopember
\end{center}

\begin{center}
  Oleh: \textbf{\name{}}
  \\NRP. \nrp{}
\end{center}

\begin{center}
  Disetujui oleh Tim Penguji Tugas Akhir:
\end{center}

\begingroup
% Menghilangkan padding
\setlength{\tabcolsep}{0pt}

\noindent
\begin{tabularx}{\textwidth}{X l}
  \advisor{}               & (Pembimbing I)   \\
  NIP: \advisornip{}       &                  \\
                           &                  \\
                           &                  \\
                           &                  \\
  \coadvisor{}             & (Pembimbing II)  \\
  NIP: \coadvisornip{}     &                  \\
                           &                  \\
                           &                  \\
                           &                  \\
  \examinerone{}           & (Penguji I)      \\
  NIP: \examineronenip{}   &                  \\
                           &                  \\
                           &                  \\
                           &                  \\
  \examinertwo{}           & (Penguji II)     \\
  NIP: \examinertwonip{}   &                  \\
                           &                  \\
                           &                  \\
                           &                  \\
                           &                  \\
                           &                  \\
                           &                  \\
\end{tabularx}
\endgroup

\begin{center}
  \textbf{\MakeUppercase{\place{}}\\\MONTH{}, \the\year{}}
\end{center}
\endgroup

\vspace*{\fill}

\cleardoublepage
\input{lainnya/pernyataan-keaslian.tex}
\cleardoublepage

% Abstrak
\chapter*{ABSTRAK}

\addcontentsline{toc}{chapter}{ABSTRAK}

\vspace{2ex}

\begin{center}
  \large\textbf{\tatitle{}}
\end{center}

\vspace{2ex}

\begingroup
% Menghilangkan padding
\setlength{\tabcolsep}{0pt}

\noindent
\begin{tabularx}{\textwidth}{l >{\centering}m{2em} X}
  \textbf{Nama / NRP}           & : & \textbf{\name{} / \nrp{}} \\
  \textbf{Departemen}           & : & \textbf{\department{}}    \\
  \textbf{Dosen Pembimbing I}   & : & \textbf{\advisor{}}       \\
  \textbf{Dosen Pembimbing II}  & : & \textbf{\coadvisor{}}     \\
\end{tabularx}
\endgroup

\noindent
\textbf{Abstrak}

% Ubah paragraf berikut dengan abstrak dari tugas akhir
Sumber daya komputasi tidak selalu digunakan.
Sumber daya komputasi terkadang berada pada posisi
\emph{idle} dan dapat sewaktu-waktu digunakan kembali.

% Pada penelitian ini kami mengajukan sistem \emph{provisioning}
% \emph{virtual cluster} yang dapat diciptakan sewaktu-waktu.

\vspace{2ex}
% Ubah kata-kata berikut dengan kata kunci dari tugas akhir
\noindent
\textbf{Kata Kunci: \emph{Multi-Tenancy}, \emph{Virtual Machine}, Kubernetes, \emph{Virtual Cluster}.}

\cleardoublepage
\chapter*{ABSTRACT}

\addcontentsline{toc}{chapter}{ABSTRACT}

\vspace{2ex}

\begin{center}
  \large\textbf{\engtatitle{}}
\end{center}

\vspace{2ex}

\begingroup
% Menghilangkan padding
\setlength{\tabcolsep}{0pt}

\noindent
\begin{tabularx}{\textwidth}{l >{\centering}m{3em} X}
  \textbf{Name / NRP} & : & \textbf{\name{} / \nrp{}} \\
  \textbf{Department} & : & \textbf{\engdepartment{}} \\
  \textbf{Advisor I}  & : & \textbf{\advisor{}}       \\
  \textbf{Advisor II} & : & \textbf{\coadvisor{}}     \\
\end{tabularx}
\endgroup

\noindent
\textbf{Abstract}

% Ubah paragraf berikut dengan abstrak dari tugas akhir dalam Bahasa Inggris
Computing resources such as computers are not always in use or in idle
position. This can lead to computing resources waste. To prevent that,
a mechanism is needed to utilize these resources  by renting out the
idle resources to users in need. One of the tools that can be used is Kubernetes.
However, Kubernetes usually requires fix amount of resources. In this study,
we propose a Kubernetes provisioning system in the form of virtual cluster. Virtual
cluster is an approach that allows computing resources to be provided on-demand. In
addition, multi-tenancy will also be implemented so that computing resources that are
rented can be used by more than one user. The result of this study is that the virtual
cluster provisioning system can be used as additional computing resources for its users
on-demand. In addition, user also shares computing resources with other users.

\vspace{2ex}
% Ubah kata-kata berikut dengan kata kunci dari tugas akhir dalam Bahasa Inggris
\noindent
\textbf{Keywords: Kubernetes, Multi-Tenancy, Virtual Cluster, Virtual Machine.}

\cleardoublepage

\begin{center}
  \Large
  \textbf{KATA PENGANTAR}
\end{center}

\addcontentsline{toc}{chapter}{KATA PENGANTAR}

\vspace{2ex}

% Ubah paragraf-paragraf berikut dengan isi dari kata pengantar

Puji dan syukur penulis panjatkan kepada Tuhan Yang Maha Esa karena atas rahmat
dan karunia-Nya, penulis dapat menyelesaikan laporan tugas akhir yang berjudul
"IMPLEMENTASI \emph{MULTI-TENANCY} UNTUK PROVISIONING KLASTER KUBERNETES" dengan
baik. Tugas akhir ini disusun sebagai persyaratan untuk memperoleh gelar sarjana
pada program studi Teknik Informatika Fakultas Teknologi Elektro dan Informatika
Cerdas Institut Teknologi Sepuluh Nopember. Penulisan tugas akhir ini tidak lepas
dari dukungan dan bantuan dari berbagai pihak.
Oleh karena itu, penulis mengucapkan terima kasih kepada:

\begin{enumerate}[nolistsep]

  \item Bapak Royyana Muslim Ijtihadie, S.Kom., M.Kom., Ph.D. selaku dosen pembimbing
    utama penulis dan Bapak Ary Mazharuddin Shiddiqi, S.Kom., M.Comp.Sc., Ph.D. selaku dosen
    pembimbing kedua penulis yang telah membantu penulis dalam menyelesaikan Tugas Akhir.

  \item Keluarga penulis, Ibu penulis, Bapak penulis dan dua Saudara penulis yang telah 
    memberikan dukungan kepada penulis. \lipsum[3][1-2]

  \item Teman-teman "\emph{Backstreet Boys}"
    
  \item Alwan Raihan, Bhatara Arundaya, Orisvio Revanda, S.T., Rayhan Rizky, dan Tigo Yoga
    selaku teman-teman "Ahmad Dahlan" yang telah memberikan semangat dan dukungan kepada
    penulis selama pengerjaan.

  \item Adam Daffa, Hanif Setyadi, Muhammad Rafi, Orisvio Revanda, S.T., dan Tio
    Widayat, S.H., selaku teman "Risin Inti Diing". Terima kasih penulis ucapkan
    sebesar-besarnya.

\end{enumerate}

Laporan tugas akhir ini jauh dari kata sempurna, oleh karena itu penulis
sangat terbuka terhadap kritik dan saran yang membangun. Akhir kata, semoga
laporan tugas akhir ini dapat memberikan dampak dan manfaat yang baik kepada
seluruh pihak yang memerlukan.

\begin{flushright}
  \begin{tabular}[b]{c}
    \place{}, \MONTH{} \the\year{} \\
    \\
    \\
    \\
    \\
    \name{}
  \end{tabular}
\end{flushright}

\cleardoublepage

% Daftar isi
\renewcommand*\contentsname{DAFTAR ISI}
\phantomsection
\addcontentsline{toc}{chapter}{\contentsname}
\tableofcontents
\cleardoublepage

% Daftar gambar
\renewcommand*\listfigurename{DAFTAR GAMBAR}
\phantomsection
\addcontentsline{toc}{chapter}{\listfigurename}
\renewcommand{\addvspace}[1]{}%
\listoffigures
\cleardoublepage

% Daftar tabel
\renewcommand*\listtablename{DAFTAR TABEL}
\phantomsection
\addcontentsline{toc}{chapter}{\listtablename}
\listoftables
\cleardoublepage

\renewcommand*\lstlistlistingname{DAFTAR KODE SUMBER DAN KODE PERINTAH}
\phantomsection
\addcontentsline{toc}{chapter}{\lstlistlistingname}
\lstlistoflistings
\cleardoublepage

% Nomor halaman isi dimulai dari sini
\pagenumbering{arabic}

% Bab 1 pendahuluan
\chapter{PENDAHULUAN}
\label{chap:pendahuluan}

% Ubah bagian-bagian berikut dengan isi dari pendahuluan

Penelitian atau implementasi ini dilatarbelakangi oleh pesatnya
penggunaan \emph{cloud computing} serta kebutuhan untuk meningkatkan
utilisasi penggunaan komputer di laboratorium Departemen Teknik Informatika
Institut Teknologi Sepuluh Nopember.

\section{Latar Belakang}
\label{sec:latarbelakang}

Pesatnya perkembangan teknologi dalam bidang komputer
seperti komputasi awan memudahkan penggunaan sumber daya
komputasi secara dinamis. Teknologi komputasi awan memungkinkan
pengguna untuk menggunakan sumber daya komputasi sesuai dengan spesifikasi
yang dibutuhkan tanpa perlu memiliki sumber daya komputasi tersebut secara
fisik. Salah satu bentuk implementasi komputasi awan adalah
Google Cloud Platform (GCP).

GCP menyediakan banyak layanan komputasi awan seperti Compute Engine yang merupakan
layanan \emph{virtual machine} yang berjalan di mesin Google dan Google Kubernetes Engine (GKE)
yang merupakan layanan Kubernetes. Semua layanan tersebut merupakan bentuk implementasi
komputasi awan karena pengguna tidak perlu memiliki sumber daya tersebut secara
fisik namun menggunakan sumber daya komputasi yang ditawarkan oleh penyedia jasa seperti GCP.

Pada tugas akhir ini, akan dikembangkan sebuah sistem \emph{provisioning} untuk virtual machine
secara dinamis yang terletak di komputer Teknik Informatika ITS. \emph{Virtual machines}
yang dibuat akan secara otomatis tergabung dalam sebuah klaster Kubernetes yang nantinya
dapat digunakan oleh pengguna. 

\section{Permasalahan}
\label{sec:permasalahan}

Dari latar belakang tersebut maka dapat dijabarkan permasalahan sebagai berikut:

\begin{enumerate}[nolistsep]

  \item Bagaimana cara mengimplementasikan \emph{multi-tenancy} pada proses \emph{provisioning}
    klaster Kubernetes?

  \item Bagaimana cara untuk membuat \emph{virtual machine} yang tergabung
    dalam sebuah \emph{cluster} secara dinamis?

\end{enumerate}

\section{Tujuan}
\label{sec:Tujuan}

Tujuan dari penelitian atau implementasi ini adalah sebagai berikut:

\begin{enumerate}[nolistsep]

  \item Membuat sistem \emph{provisioning} yang dapat membuat \emph{virtual machine}
    secara dinamis sesuai dengan kriteria \emph{resource} permintaan pengguna.

\end{enumerate}

\section{Batasan Masalah}
\label{sec:batasanmasalah}

Batasan-batasan dari penelitian atau implementasi adalah sebagai berikut:

\begin{enumerate}[nolistsep]

  \item \emph{Provisioning} terjadi di lingkungan Teknik Informatika Institut
    Teknologi Sepuluh Nopember

\end{enumerate}

\section{Sistematika Penulisan}
\label{sec:sistematikapenulisan}

Laporan penelitian tugas akhir ini terbagi menjadi beberapa bagian seperti
berikut:

\begin{enumerate}[nolistsep]

  \item \textbf{BAB I Pendahuluan}

        Bab ini berisi pengantar sekaligus gambaran tugas akhir
        secara garis besar. Bab ini terdiri dari lima subbab, yaitu subbab
        latar belakang, subbab permasalahan, subbab tujuan, subbab batasan
        masalah, dan subbab sistematika penulisan.

        \vspace{2ex}

  \item \textbf{BAB II Tinjauan Pustaka}

        Bab ini berisi tinjauan pustaka dari teknologi-teknologi
        yang telah diciptakan sebelumnya untuk membantu dalam proses
        implementasi.

        \vspace{2ex}

  \item \textbf{BAB III Desain dan Implementasi Sistem}

        Bab ini berisi penjelasan desain dan implementasi dari sistem.

        \vspace{2ex}

  \item \textbf{BAB IV Pengujian dan Analisa}

        Bab ini berisi pengujian dan analisa dari sistem yang telah dibuat.
        Pengujian dan analisa dilakukan untuk mengetahui apakah sistem yang
        telah dibuat ... .

        \vspace{2ex}

  \item \textbf{BAB V Penutup}

        Bab ini berisi

\end{enumerate}

\cleardoublepage

% Bab 2 tinjauan pustaka
\chapter{TINJAUAN PUSTAKA}
\label{chap:tinjauanpustaka}

% Ubah bagian-bagian berikut dengan isi dari tinjauan pustaka

Demi mendukung implementasi pada tugas akhir, dasar teori
yang berkaitan akan dijelaskan pada bab ini.
Kajian yang didapat dan dasar teori tersebut
akan digunakan pada tugas akhir ini.

\section{Kubernetes}
\label{sec:kubernetes}

Kubernetes adalah sebuah platform manajemen aplikasi yang dikemas (\emph{containerized applications})
bersifat sumber terbuka. Kubernetes dibuat berdasarkan \emph{tool} internal yang
diciptakan oleh Google bernama Borg untuk mengelola layanan mereka dalam bentuk \emph{container},
kemudian beberapa pengembang Borg menciptakan platform serupa
namun bersifat sumber terbuka yang diberi nama Kubernetes \parencite{borg-references}.

Kubernetes dapat mengelola jalannya aplikasi berbasis kontainer seperti membuat
aplikasi yang tersebut \emph{scalable} dengan cara menambah atau mengurangi \emph{container}
yang menjalankan aplikasi tersebut sesuai dengan kebutuhan pengguna. Untuk memenuhi
hal tersebut, Kubernetes menyediakan beberapa fitur seperti berikut:

\begin{enumerate}
  \item{\emph{Service discovery} dan \emph{load balancing}}
    \par{\emph{Container} pada Kubernetes diekspos oleh Kubernetes menggunakan DNS atau IP dari
      \emph{container} tersebut. Jika \emph{traffic} menuju \emph{container} tersebut tinggi, Kubernetes
      dapat menggunakan \emph{load balancer} agar \emph{deployment} tetap stabil.
    }
  \item{Orkestrasi penyimpanan}
    \par{Kubernetes dapat menggunakan beberapa jenis sistem penyimpanan, seperti
      penyimpanan lokal, penyimpanan dari penyedia jasa \emph{cloud}, dan sebagainya
    }
  \item{\emph{Rollouts} dan \emph{rollbacks} secara otomatis}
    \par{Kubernetes dapat menyesuaikan \emph{state} dari konfigurasi \emph{state} yang
      diberikan, sehingga status dari \emph{state} klaster Kubernetes akan selalu mengikuti
      \emph{state} yang diberikan
    }
  \item{\emph{Self-healing}}
    \par{Kubernetes akan memulai ulang \emph{container} yang gagal atau mati dalam
      mengerjakan \emph{job}, menggantikan \emph{container}, dan akan selalu menunggu
      \emph{container} yang belum siap sebelum pengguna dapat menggunakannya
    }
  \item{\emph{Horizontal scaling}}
    \par{Aplikasi yang di-\emph{deploy} pada Kubernetes dapat di-\emph{scaling} secara
      \emph{horizontal}, yaitu penambahan Pod untuk pengerjaan aplikasi tersebut
    }
  \item{IPv4 dan IPv6 \emph{dual-stack}}
    \par{Kubernetes dapat menggunakan IPv4 dan IPv6 untuk alamat IP dari objek dalam Kubernetes
    }
  \item{Desain untuk ekstensibilitas}
    \par{Kubernetes didesain agar mudah dieksten tanpa merubah \emph{source code} dari Kubernetes
    }
\end{enumerate}

\subsection{Arsitektur Kubernetes}

Kubernetes memiliki konsep klaster yang merupakan kumpulan dari satu atau
lebih mesin. Klaster Kubernetes terdiri dari \emph{control plane} yang mengatur
mengatur \emph{worker nodes} dan Pod pada klaster
beserta kumpulan mesin \emph{worker} yang disebut sebagai \emph{worker nodes} yang 
menjalankan aplikasi \emph{containerized}. Setiap \emph{cluster} membutuhkan
setidaknya satu \emph{worker nodes} untuk menjalankan Pod. Arsitektur
klaster Kubernetes dapat dilihat pada gambar \ref{fig:arsitektur-cluster-kubernetes}

\begin{figure}[H]
  \centering

  % Ubah dengan nama file gambar dan ukuran yang akan digunakan
  \includegraphics[scale=0.2]{gambar/kubernetes-cluster-architecture.png}

  % Ubah dengan keterangan gambar yang diinginkan
  \caption{Arsitektur klaster Kubernetes}
  \label{fig:arsitektur-cluster-kubernetes}
\end{figure}

\emph{Cluster} tersebut adalah tempat dimana semua \emph{containerized applications},
\emph{jobs}, dan komponen Kubernetes berjalan. 

Untuk membuat \emph{cluster}, Kubernetes memiliki \emph{tool} berupa kubeadm

\subsubsection{Kubernetes Control Plane}

Pada klaster Kubernetes, mesin yang menjadi \emph{control plane} bertugas
untuk membuat keputusan global seperti \emph{scheduling} serta mendeteksi
dan merespon \emph{event} yang terjadi pada klaster tersebut. Untuk melakukan
tugas tersebut, \emph{control plane} memiliki beberapa komponen yaitu kube-apiserver,
etcd, kube-scheduler, kube-controller-manager, dan cloud-controller-manager.

\begin{enumerate}
  
  \item Komponen kube-apiserver berperan sebagai antarmuka dari \emph{control plane}.
    Komponen kube-apiserver melayani komunikasi menggunakan operasi RESTful dan
    mengekspos \emph{shared state} dari klaster yang nantinya digunakan untuk
    interaksi antar komponen.

  \item Komponen etcd merupakan tempat penyimpanan \emph{key-value} secara konsisten.
    Komponen etcd digunakan untuk menyimpan semua data pada klaster seperti \emph{state}
    klaster saat ini dan \emph{state} yang diinginkan.

  \item Komponen kube-scheduler merupakan komponen yang mengamati Pod yang telah
    selesai dibuat dan menempatkan Pod ke \emph{node} untuk dijalankan. Pada saat
    menempatkan Pod ke \emph{node}, kube-scheduler memperhitungkan faktor-faktor
    yang berpengaruh seperti batasan \emph{hardware/software}, kebutuhan sumber daya secara
    individu dan kolektif, spesifikasi afinitas dan non-afinitas, \emph{data locality}, interferensi
    antar beban kerja, dan tenggat waktu dari beban kerja.

  \item Komponen kube-controller-manager berperan untuk menjalankan proses \emph{controller}.
    Proses \emph{controller} mengatur komponen sesuai dengan jenis proses \emph{controller},
    salah satu contohnya adalah proses \emph{node controller} yang mengatur jalannya \emph{node}
    seperti mengamati \emph{node} dan merespon ketika \emph{node} mengalami kegagalan.

  \item Komponen cloud-controller-manager merupakan komponen yang berperan sebagai penghubung
    klaster dengan API khusus dari penyedia layanan awan. Komponen ini memisah
    komponen yang berinteraksi dengan penyedia layanan dan komponen yang hanya berinteraksi dengan
    klaster.

\end{enumerate}

\subsubsection{Kubernetes Node}

Pada sebuah klaster Kubernetes, mesin atau server selain \emph{control plane} akan menjadi \emph{node}.
\emph{Node} bertugas untuk menjalankan Pod. Untuk membantu dalam menjalankan tugas tersebut, \emph{node}
memiliki beberapa komponen yaitu kubelet, \emph{container runtime}, dan kube-proxy.

\begin{enumerate}
  
  \item Komponen kubelet merupakan sebuah komponen untuk memastikan \emph{container} berjalan di dalam Pod.
    Komponen kubelet menggunakan konfigurasi dari Podspec yang berupa objek YAML atau JSON yang mendeskripsikan
    sebuah Pod. Komponen kubelet menerima konfigurasi Podspec yang disediakan dari banyak mekanisme seperti
    apiserver dan memastikan bahwa \emph{container} yang didefinisikan pada Podspec berjalan dengan baik. Komponen kubelet
    tidak mengelola \emph{container} yang tidak diciptakan oleh Kubernetes.

  \item Komponen \emph{container runtime} mengelola \emph{container} pada Kubernetes. Komponen \emph{container runtime}
    bertanggung jawab untuk mengelola \emph{container} seperti eksekusi dan siklus hidup dari \emph{container} yang berjalan
    di lingkungan Kubernetes. Kubernetes mendukung \emph{container runtime} yang mengimplementasikan Kubernetes
    CRI (\emph{Container Runtime Interface}).

  \item Komponen kube-proxy mengatur pengaturan jaringan pada setiap \emph{node}. Pengaturan tersebut mengatur komunikasi
    jaringan menuju Pod dari dalam atau luar klaster. Kube-proxy memastikan \emph{request} diteruskan menuju ke
    \emph{container} yang tepat.

\end{enumerate}

\subsection{Objek pada Kubernetes}

Objek pada Kubernetes adalah entitas persisten pada sistem Kubernetes dan digunakan untuk
merepresentasikan keadaan dalam klaster tersebut. Keadaan dalam klaster yang direpresentasikan
oleh objek-objek pada Kubernetes adalah aplikasi yang sedang berjalan pada klaster, \emph{node}
tempat aplikasi berjalan, sumber daya yang tersedia, serta pengaturan kebijakan tentang aplikasi
tersebut.

\subsubsection{Pod}

Pod merupakan unit terkecil yang dapat dibuat dan dikelola pada klaster Kubernetes. Pod sendiri adalah
kumpulan dari satu atau lebih \emph{container} yang berbagi alamat IP dan volume serta
pengaturan mengenai bagaimana \emph{container} tersebut dijalankan. Konfigurasi Pod dapat
diatur dengan konfigurasi deklaratif menggunakan Podspec yang memiliki ekstensi YAML atau JSON.

\subsubsection{Replication Controllers dan ReplicaSet}

Replication Controllers merupakan objek yang mendefinisikan sebuah \emph{pod template}
dan parameter kontrol untuk melakukan \emph{scaling} secara horizontal, yaitu dengan
cara menambahkan atau mengurangi jumlah dari Pod yang sedang berjalan. Replication
Controllers bertanggung jawab untuk memastikan jumlah Pod yang ada pada klaster
sesuai dengan jumlah Pod yang didefinisikan pada konfigurasi.

\subsubsection{Job}

Job merupakan beban kerja pada Kubernetes yang berbasis tugas. Job akan menciptakan
satu atau lebih Pod untuk menyelesaikan tugas Job tersebut. Ketika tugas tersebut
selesai, Pod yang menjalankan tugas tersebut akan berhenti. Job menjamin jumlah
Pod yang diciptakan akan selalu sesuai dengan jumlah Pod yang diinginkan pada
saat menjalankan tugas Job. Ketika sebuah Pod mengalami kegagalan dalam menjalankan
tugas Job, Job dapat menciptakan Pod baru untuk menyelesaikan tugasnya. Job dapat
menciptakan lebih dari satu Pod untuk menjalankan tugas secara paralel.

\subsubsection{Service}

Service merupakan abstraksi dari Pod yang menyediakan alamat IP dan DNS yang
akan digunakan untuk mengakses Pod tersebut. Service pada Kubernetes merupakan
objek yang mengimplementasikan arsitektur REST. Terdapat beberapa tipe dari objek
Service, yaitu ClusterIP, NodePort, LoadBalancer, dan ExternalName.

\begin{enumerate}
  
  \item{ClusterIP merupakan tipe dari Service yang melakukan ekspos Service pada.}

  \item{
      NodePort merupakan tipe dari Service yang melakukan ekspos Service pada alamat IP
      dan \emph{port} statik setiap \emph{node}. NodePort memungkinkan Service untuk diakses dari luar
      dengan melakukan \emph{request} pada NodeIP dan NodePort.
    }

  \item{
      LoadBalancer merupakan tipe dari Service 
    }

  \item{
      ExternalName merupakan tipe dari Service yang memetakan Service dengan \emph{externalName}
      dengan cara mengembalikan catata CNAME bersama dengan nilainya.
    }

\end{enumerate}

\subsection{K3s}

Kubernetes memiliki banyak komponen untuk menjalankan platform Kubernetes. Banyak komponen
tersebut terkadang tidak diperlukan di beberapa skenario. 

K3s merupakan distribusi Kubernetes yang diciptakan oleh Rancher yang
memiliki \emph{size} yang lebih kecil secara keseluruhan. K3s berbentuk
\emph{binary} yang berisi semua \emph{tools} yang dibutuhkan untuk membuat
\emph{cluster} dan bergabung ke \emph{cluster}.

\section{\emph{Multi-tenancy}}
\label{sec:multi-tenancy}

\emph{Multi-tenancy} secara bahasa memiliki arti banyak pengguna atau \emph{tenant}. Dalam konteks
\emph{cloud computing}, \emph{multi tenancy} memiliki definisi pembagian \emph{resource} 
untuk setiap pengguna namun definisi tersebut masih bersifat luas karena implementasi
dari \emph{multi-tenancy} berbeda untuk setiap \emph{service models} \parencite{6830928}.

\section{\emph{Virtual Machine}}
\label{sec:virtual-machine}

\emph{Virtual machine} merupakan lingkungan komputasi terisolasi
yang berjalan di atas mesin fisik. Digagaskan pertama kali oleh IBM
dengan sebuah konsep bernama \emph{time-sharing}. \emph{Time-sharing}
memungkinkan sebuah mesin digunakan oleh lebih dari satu pengguna sehingga
pengguna tersebut terlihat mempunyai sebuah mesin tersendiri.

\begin{figure}[H]
  \centering

  % Ubah dengan nama file gambar dan ukuran yang akan digunakan
  \includegraphics[scale=0.5]{gambar/virtual-machine-architecture.png}

  % Ubah dengan keterangan gambar yang diinginkan
  \caption{Arsitektur \emph{virtual machine}}
  \label{fig:arsitektur-virtual-machine}
\end{figure}

\subsection{QEMU}

QEMU adalah emulator dan virtualisasi mesin generik dan sumber terbuka.

\section{Libvirt}
\label{sec:libvirt}

Libvirt merupakan API untuk berkomunikasi dengan teknologi virtualisasi yang disediakan
oleh sistem operasi.

\section{\emph{Network Bridge}}
\label{sec:network-bridge}

NKASNDLKASNLDASNK

\section{gRPC}

\section{Roket Luar Angkasa}
\label{sec:roketluarangkasa}

% Contoh input gambar
\begin{figure}[H]
  \centering

  % Ubah dengan nama file gambar dan ukuran yang akan digunakan
  \includegraphics[scale=0.35]{gambar/roketluarangkasa.jpg}

  % Ubah dengan keterangan gambar yang diinginkan
  \caption{Peluncuran roket luar angkasa \emph{Discovery} \parencite{roketluarangkasa}.}
  \label{fig:roketluarangkasa}
\end{figure}

Roket luar angkasa merupakan \lipsum[1]

\emph{Discovery}, Gambar \ref{fig:roketluarangkasa}, merupakan \lipsum[2]

\subsection{Hukum Newton}
\label{subsec:hukumnewton}

Newton \parencite{newton1687} pernah merumuskan bahwa \lipsum[1]
Kemudian menjadi persamaan seperti pada persamaan \ref{eq:hukumpertamanewton}.

% Contoh pembuatan persamaan
\begin{equation}
  \label{eq:hukumpertamanewton}
  \sum \mathbf{F} = 0\; \Leftrightarrow\; \frac{\mathrm{d} \mathbf{v} }{\mathrm{d}t} = 0.
\end{equation}

\cleardoublepage

% Bab 3 desain dan implementasi
\chapter{METODOLOGI}
\label{chap:metodologi}

Rancangan implementasi \emph{multi-tenancy} untuk \emph{provisioning}
klaster Kubernetes dimulai dari membuat aplikasi \emph{service} untuk \emph{provisioning}
yang terletak di komputer \emph{worker} untuk membuat
\emph{virtual machine}, kemudian membuat situs web \emph{dashboard} yang nantinya
digunakan oleh \emph{user} untuk membuat klaster Kubernetes secara dinamis dan sewaktu-waktu
menggunakan \emph{virtual machine} yang dibuat oleh aplikasi \emph{provisioning}.
Rancangan tahapan penyelesain implementasi tugas akhir ini
dapat dilihat pada gambar \ref{fig:top-level-implementation}.

\begin{figure}[H]
  \centering
  \includegraphics[scale=0.6]{gambar/top-level-implementation.png}
  \caption{Rancangan Penyelesaian Implementasi Tugas Akhir}
  \label{fig:top-level-implementation}
\end{figure}

Berdasarkan gambar \ref{fig:top-level-implementation} di atas, tahapan pertama
dalam rancangan implementasi tugas akhir adalah perancangan arsitektur sistem.
Tahapan perancangan arsitektur sistem berdasarkan kebutuhan sistem yang diperlukan
yaitu situs web sebagai \emph{dashboard} dan aplikasi \emph{provisioning}
untuk membuat \emph{virtual machine}. Setelah perancangan arsitektur sistem telah
selesai, implementasi dari rancangan sebelumnya akan dilakukan. Tahap terakhir
dari tugas akhir ini adalah uji coba dan evaluasi dari sistem yang telah dirancang.
Uji coba dan evaluasi ini bertujuan untuk memeriksa apakah sistem yang telah
diimplementasikan dapat berjalan sesuai kebutuhan dari tugas akhir ini.

\section{Perancangan Arsitektur Sistem}
\label{sec:perancanganarsitektursistem}

Secara garis besar, \emph{workflow} dari implementasi tugas akhir ini dapat dilihat
pada gambar \ref{fig:website-flowchart}. Selain itu, gambaran dari arsitektur sistem
implementasi tugas akhir ini terdapat
pada gambar \ref{fig:server-worker-top-level}. Berdasarkan gambar \ref{fig:server-worker-top-level},
komunikasi antara \emph{worker} dan server utama adalah komunikasi dua arah. Server utama
berkomunikasi dengan \emph{worker} untuk membuat \emph{virtual machine} dan \emph{worker}
berkomunikasi dengan server utama untuk memberi status bahwa \emph{worker} sudah siap
dalam menerima permintaan untuk membuat \emph{virtual machine} serta memberi status dari
pembuatan \emph{virtual machine}.

Setiap komputer fisik yang akan dijadikan \emph{worker} memerlukan aplikasi
atau sebuah \emph{service} yang dapat menerima permintaan pembuatan 
\emph{virtual machine} yang dikirimkan melalui server utama. Selain itu,
server utama juga memerlukan aplikasi yang dapat digunakan oleh pengguna
untuk membuat klaster dengan \emph{virtual machine} pada \emph{worker}.

\begin{figure}[H]
  \centering
  \includegraphics[scale=0.6]{gambar/flowchart-website.png}
  \caption{\emph{Workflow} Penggunaan Aplikasi}
  \label{fig:website-flowchart}
\end{figure}

\begin{figure}[H]
  \centering
  \includegraphics[scale=0.6]{gambar/server-worker-top-level.png}
  \caption{Gambaran Arsitektur Sistem}
  \label{fig:server-worker-top-level}
\end{figure}

\section{Pengembangan Sistem \emph{Provisioning}}
\label{sec:implementasi-sistem-provisioning}

Sistem \emph{provisioning} pada setiap komputer \emph{worker} secara fungsi dapat dipisah
menjadi dua bagian besar, yaitu bagian server untuk menerimaa permintaan pembuatan klaster yang dikirim
oleh server utama dan bagian pembuatan \emph{virtual machine}.

\subsection{Pengembangan Server}
\label{sec:server}

Komputer \emph{worker} menerima \emph{request} pembuatan klaster melalui jaringan internet.
Oleh karena itu, \emph{worker} dapat menerima \emph{request} dengan membuka dan mendengarkan \emph{port}
untuk menerima \emph{request} dari server utama. Implementasi tugas akhir ini menggunakan protokol
RPC dengan gRPC untuk menerima \emph{request} tersebut.

\begin{lstlisting}[
  caption={\emph{Pseudocode} Definisi Prosedur RPC pada Komputer \emph{Worker} Bagian 1},
  label={lst:rpc-procedure-1}
]
PSEUDOCODE_definisi prosedur RPC pada komputer worker

procedure {
  CreateMaster(CreateMasterRequest) returns (CreateMasterResponse) {}
  CreateWorker(CreateWorkerRequest) returns (CreateWorkerResponse) {}
}
\end{lstlisting}

\begin{lstlisting}[
  caption={Definisi Prosedur RPC pada Komputer \emph{Worker} Bagian 2},
  label={lst:rpc-procedure-2}
]
message CreateMasterRequest {
  string cluster_name
  string cluster_token
  number cpu, storage, memory
}

message CreateMasterResponse {
  bool is_success
  string message
  string master_ip_address
  string dashboard_token
}

message CreateWorkerRequest {
  string cluster_name
  string cluster_token
  string master_ip_address
  number cpu, storage, memory
}

message CreateMasterResponse {
  bool is_success
  string message
  string ip_address
  string dashboard_token
}
\end{lstlisting}

Kode sumber \ref{lst:rpc-procedure-1} dan \ref{lst:rpc-procedure-2} menunjukkan fungsi atau prosedur
yang disediakan oleh \emph{worker} menggunakan protokol RPC. Untuk membuat prosedur
di gRPC, format yang dipakai adalah format Protobuf. Protobuf merupakan format
yang digunakan oleh gRPC sebagai \emph{Interface Definition Language}.

Prosedur yang sudah ditulis menggunakan Protobuf akan dikonversi
ke bahasa pemrograman yang diinginkan. Hasil konversi tersebut hanya
berupa fungsi atau prosedur tanpa isi. Isi atau \emph{logic} dari fungsi atau prosedur
tersebut harus dibuat sesuai dengan keinginan pengguna. Hasil konversi dari Protobuf
dapat dibagikan ke komputer lainnya yang memerlukan prosedur tersebut. 

Pada implementasi tugas akhir ini, komputer \emph{worker} menerima
hasil konversi dari prosedur dari Protobuf dan membuat isi dari
prosedur yang sudah ditetapkan. Server utama juga menerima hasil konversi
dan akan menggunakannya untuk menjalankan fungsi atau prosedur tersebut di komputer
\emph{worker}. Isi dari prosedur yang akan dijalankan di komputer \emph{worker} terdapat
pada \emph{pseudocode} \ref{lst:isi-prosedur-1} dan \emph{pseudocode} \ref{lst:isi-prosedur-1}.
Isi prosedur tersebut adalah \emph{logic} untuk membuat \emph{virtual machine} yang berjenis
\emph{master node} atau \emph{worker node}. Perbedaan dari keduanya adalah \emph{master node}
akan bertugas menjadi \emph{control plane} dari klaster Kubernetes yang dibuat nantinya.

\clearpage

\begin{lstlisting}[
  caption={\emph{Pseudocode} Pembuatan \emph{Control Plane} Pada Komputer \emph{Worker}},
  label={lst:isi-prosedur-1}
]
PSEUDOCODE_pembuatan control plane

FUNC CreateMaster
PARAMETER CreateMasterRequest
DO
    LET response BE CreateMasterResponse
    LET instanceRequest BE CreateInstanceRequest
    SET instanceRequest TO CreateMasterRequest

    ADD instanceRequest TO redis.queue
    CALL redis.subscribe WITH request
    
    CASE resultChannel OF
    value:
        SET response TO resultChannel.value
        RETURN response
    ELSE
        RETURN ERROR timeoutExceeded
DONE
\end{lstlisting}

\begin{lstlisting}[
  caption={\emph{Pseudocode} Pembuatan \emph{Worker Node} Pada Komputer \emph{Worker}},
  label={lst:isi-prosedur-2}
]
PSEUDOCODE_pembuatan worker node

FUNC CreateWorker
PARAMETER CreateWorkerRequest
RETURN CreateWorkerResponse 
DO
    LET response BE CreateWorkerResponse
    LET instanceRequest BE CreateInstanceRequest
    SET instanceRequest TO CreateWorkerRequest

    ADD instanceRequest TO redis.queue
    CALL redis.subscribe WITH request
    
    CASE resultChannel OF
    value:
        SET response TO resultChannel.value
        RETURN response
    OTHERS:
        RETURN ERROR timeoutExceeded
DONE
\end{lstlisting}

Komputer \emph{worker} perlu mendengarkan \emph{request} pembuatan melalui
protokol gRPC yang sudah ditetapkan sebelumnya. \emph{Library} gRPC pada Golang
menyediakan API untuk server untuk mendengarkan \emph{request} gRPC di \emph{port}
tertentu. Setelah server mendengarkan di \emph{port} yang telah ditentukan, maka
server tersebut dapat menerima \emph{request} RPC dengan prosedur atau fungsi
yang sudah dibuat melalui \emph{stub}. Kode sumber untuk menjalankan
gRPC server dapat dilihat pada kode sumber \ref{lst:rpc-server-start}.

\clearpage

\begin{lstlisting}[
  caption={Memulai gRPC Server},
  label={lst:rpc-server-start},
]
PSEUDOCODE_menjalankan gRPC server

FUNC StartGrpcServer
DO
    LET listener BE TCP_LISTENER(port)
    LET grpcServer BE GRPC_SERVER

    CALL start_worker
    CALL connect_to_main_server
    CALL grpcServer.serve WITH listener
DONE
\end{lstlisting}

\subsection{Pengembangan Sistem \emph{Provisioning}}
\label{sec:provisioning}

Setelah komputer menerima \emph{request} pembuatan klaster melalui RPC,
\emph{request} tersebut diteruskan ke bagian \emph{provisioning}. Alur 
implementasi \emph{provisioning} dari implementasi ini adalah mempersiapkan
sistem operasi Linux yang akan digunakan sebagai sistem operasi dari
\emph{virtual machine}, mempersiapkan Linux \emph{bridge},
implementasi QEMU dan Libvirt untuk membuat \emph{virtual machine},
implementasi Cloud-init untuk mengkustomisasi \emph{virtual machine}, 
serta implementasi \emph{queue} dan \emph{worker}.
% dan implementasi Websocket untuk \emph{logging}.

\subsubsection{Persiapan Linux untuk \emph{Virtual Machine}}
\label{sec:persiapan-linux-untuk-virtual-machine}

Dalam mempersiapkan sistem operasi Linux yang akan dipakai, terdapat beberapa
hal yang harus dipertimbangkan. Karena \emph{virtual machine} yang akan digunakan
harus siap dengan konfigurasi Cloud-init, maka Linux yang digunakan bukan Linux
yang berbentuk ISO. Linux berbentuk ISO pada saat \emph{booting} akan mengarahkan
\emph{user} untuk proses instalasi sistem operasi Linux dan proses tersebut memerlukan
interaksi dengan pengguna. Selain itu, Linux dalam bentuk ISO tidak memiliki Cloud-init
sehingga tidak memungkinkan untuk menjalankan proses Cloud-init.

Untuk memenuhi kebutuhan tersebut, maka \emph{image} Linux yang digunakan
adalah \emph{image} yang khusus untuk kebutuhan \emph{cloud}. Ubuntu merupakan
salah satu distribusi Linux yang menyediakan \emph{image} khusus tersebut.
Pada implementasi tugas akhir ini, versi Ubuntu yang digunakan adalah Ubuntu 24.10
dengan nama kode "Oracular Oriole". Jenis \emph{file} dari \emph{image} yang akan digunakan
adalah \emph{file} dengan ekstensi .img yang merupakan format \emph{file} untuk \emph{disk image}
yang digunakan oleh QEMU.

Versi Ubuntu yang digunakan pada \emph{cloud image} adalah versi Ubuntu Server yang
berukuran lebih kecil dari \emph{desktop image} dan berisi aplikasi dan \emph{tools}
yang sering digunakan pada server. Namun pada Ubuntu \emph{cloud image}, tidak ada
kata sandi untuk bisa \emph{login} ke \emph{root} pada sistem operasi. Untuk bisa \emph{login},
dibutuhkan kata sandi untuk \emph{root} pada \emph{cloud image} tersebut. Salah satu \emph{tool} yang
dapat digunakan adalah \lstinline{virt-customize}.
Untuk menambahkan kata sandi untuk \emph{root} menggunakan \lstinline{virt-customize} adalah
sebagai berikut:

\begin{lstlisting}[
  style=clistyle,
  caption={\emph{Command} Linux untuk Konfigurasi \emph{Cloud Image}}
]
# virt-customize --add oracular-server-cloudimg-amd64.img --root-password password:root
\end{lstlisting}

Pada \emph{command} di atas, \lstinline{oracular-server-cloudimg-amd64.img} menandakan \emph{file disk image}
mana yang ingin dimodifikasi, \lstinline{--root-password password:root} menandakan bahwa kata sandi
dari \emph{user root} adalah root.

Setelah \emph{user root} sudah memiliki kata sandi, \emph{user root} dapat digunakan
pada Ubuntu tersebut. Pada implementasi tugas akhir ini, modifikasi lain yang dilakukan
terhadap \emph{cloud image} ini adalah mengunduh \emph{binary} dari K3s dan Helm untuk mengurangi
waktu yang dibutuhkan untuk membuat \emph{virtual machine}. Pada implementasi tugas akhir ini,
versi K3s yang digunakan adalah v1.33.1+k3s1 (99d91538) dan versi Helm yang digunakan adalah
v3.18.1 dengan \emph{commit} terakhir f6f87.

\begin{figure}[H]
  \centering
  \includegraphics[scale=0.4]{gambar/k3s-helm-version.png}
  \caption{Versi K3s dan Helm}
  \label{fig:k3s-helm-version}
\end{figure}

\emph{Cloud image} Ubuntu secara \emph{default} sudah memiliki Cloud-init. Cloud-init tersebut
tidak perlu dikonfigurasi untuk dapat digunakan. Pada saat \emph{booting virtual machine} untuk yang
pertama kali dengan menggunakan \emph{file} konfigurasi yang diperlukan, Cloud-init akan secara otomatis
menjalankan konfigurasi dari \emph{file} tersebut.

\subsubsection{Persiapan Linux \emph{Bridge}}
\label{sec:persiapan-linux-bridge}

Proses penggabungan \emph{worker node} menggunakan K3s memerlukan alamat IP
dari \emph{control plane}. Karena \emph{virtual machine} yang bertugas menjadi
\emph{worker node} tidak selalu berada di komputer fisik yang sama, sehingga
\emph{control plane} dan \emph{worker node} memerlukan alamat IP di jangkauan
yang sama dengan komputer \emph{host} agar dapat berkomunikasi satu sama lain.
\emph{Virtual machine} dapat memiliki alamat IP di jangkauan yang sama dengan
alamat IP dari \emph{host} menggunakan Linux \emph{bridge}.

Untuk membuat Linux \emph{bridge} di sistem operasi Linux pada implementasi tugas akhir ini,
\emph{tools} NetworkManager akan digunakan. Pemilihan \emph{tools} tersebut karena
NetworkManager merupakan \emph{default tools} yang digunakan pada distribusi Linux
Ubuntu. Selain itu, NetworkManager juga dapat mengatur koneksi internet pada
komputer, termasuk pembuatan Linux \emph{bridge}. Untuk membuat Linux \emph{bridge}
yang bernama k3s-br0 dan \emph{network interface} dari ethernet yang digunakan
adalah enp1s0, dapat menggunakan \emph{command line} sebagai berikut:

\begin{lstlisting}[
  style=clistyle,
  caption={\emph{Command} Linux untuk Konfigurasi Linux \emph{Bridge}},
]
# nmcli con add type bridge ifname k3s-br0 con-name k3s-br0
# nmcli con add type bridge-slave ifname enp1s0 master k3s-br0
# nmcli con down enp1s0
# nmcli con up k3s-br0
\end{lstlisting}

Setelah itu, komputer akan tetap tersambung dengan koneksi internet melalui
\emph{network interface} enp1s0 dan Linux \emph{bridge} k3s-br0 juga aktif.
Linux \emph{bridge} k3s-br0 akan digunakan pada saat membuat \emph{virtual machine}.
\subsubsection{Penggunaan Cloud-init}
\label{sec:implementasi-cloud-init}

Konfigurasi Cloud-init yang digunakan untuk \emph{virtual machine} yang bertugas sebagai \emph{control plane}
dan \emph{worker node} berbeda. Untuk \emph{virtual machine} yang bertugas menjadi \emph{control plane},
ada konfigurasi tambahan dibandingkan dengan \emph{virtual machine} yang bertugas menjadi \emph{worker node}.

\emph{Control plane} dan \emph{worker node} menggunakan konfigurasi jaringan yang sama. \emph{Cloud image}
dari Ubuntu menggunakan \emph{network interface} enp1s0 sebagai \emph{network interface} untuk
koneksi dengan ethernet. Oleh karena itu, \emph{control plane} dan \emph{worker node} menggunakan
konfigurasi jaringan yang sama karena menggunakan \emph{cloud image} yang sama. Kode sumber untuk
konfigurasi jaringan melalui Cloud-init terdapat pada kode sumber \ref{lst:konfigurasi-jaringan}.

\lstinputlisting[
  caption={Konfigurasi Jaringan},
  label={lst:konfigurasi-jaringan},
  style=codestyle,
]{program/cloud-init-network-configuration.yaml}

Pada kode sumber \ref{lst:konfigurasi-jaringan}, sintaks konfigurasi yang dipakai adalah sintaks
konfigurasi versi 2 dari Cloud-init. \emph{Network interface} yang dikonfigurasi adalah \emph{network interface}
enp1s0. \emph{Network interface} tersebut merupakan \emph{default network interface} yang digunakan
oleh \emph{cloud image} Ubuntu untuk jaringan ethernet. Protokol dhcp digunakan agar \emph{virtual machine} mendapatkan
IP secara dinamis dari \emph{host} yang menyediakan Linux \emph{bridge}.

Setelah melakukan konfigurasi jaringan, akan dilakukan konfigurasi lainnya.
Pada \emph{control plane}, konfigurasi lanjutan dari Cloud-init dapat dilihat 
pada kode sumber \ref{lst:konfigurasi-lanjutan-control-plane}.
Untuk \emph{worker node}, konfigurasi lanjutan dari Cloud-init dapat dilihat
pada kode sumber

\begin{lstlisting}[
  caption={Konfigurasi Lanjutan pada \emph{Control Plane} Bagian 1},
  label={lst:konfigurasi-lanjutan-control-plane},
  style=codestyle,
]
#cloud-config
hostname: %s
locale: en_US.UTF-8
timezone: Asia/Jakarta
users:
- default
- name: user
  groups: sudo
  sudo: ALL=(ALL:ALL) ALL
  plain_text_passwd: user
  lock_passwd: false
  shell: /bin/bash

write_files:
- path: /root/service-account.yaml
  content: |
    apiVersion: v1
    kind: ServiceAccount
    metadata:
      name: admin-user
      namespace: kubernetes-dashboard
\end{lstlisting}

\clearpage

\begin{lstlisting}[
  caption={Konfigurasi Lanjutan pada \emph{Control Plane} Bagian 2},
  label={lst:konfigurasi-lanjutan-control-plane-2},
  style=codestyle,
]
- path: /root/role-binding.yaml
  content: |
    apiVersion: rbac.authorization.k8s.io/v1
    kind: ClusterRoleBinding
    metadata:
      name: admin-user
    roleRef:
      apiGroup: rbac.authorization.k8s.io
      kind: ClusterRole
      name: cluster-admin
    subjects:
    - kind: ServiceAccount
      name: admin-user
      namespace: kubernetes-dashboard
- path: /etc/systemd/system/kube-dashboard.service
  content: |
    [Unit]
    Description=Kubernetes dashboard
    Wants=network-online.target
    After=k3s.service

    [Install]
    WantedBy=multi-user.target

    [Service]
    Type=simple
    User=root
    Restart=always
    RestartSec=5s
    ExecStart=/usr/local/bin/k3s \
        kubectl -n kubernetes-dashboard \
        port-forward svc/kubernetes-dashboard-kong-proxy \
        8443:443 --address 0.0.0.0 \

runcmd:
- |
  echo "running command"
  echo "updating and upgrading packages"
  echo "installing necessary packages"

  echo "installing k3s"
  curl -sfL https://get.k3s.io | INSTALL_K3S_SKIP_DOWNLOAD=true INSTALL_K3S_EXEC="server --token %s" sh -s -

  export KUBECONFIG=/etc/rancher/k3s/k3s.yaml
  echo "export KUBECONFIG=/etc/rancher/k3s/k3s.yaml" >> /etc/profile

  echo "creating kubernetes dashboard"
  helm repo add kubernetes-dashboard https://kubernetes.github.io/dashboard/
\end{lstlisting}

\clearpage

\begin{lstlisting}[
  caption={Konfigurasi Lanjutan pada \emph{Control Plane} Bagian 3},
  label={lst:konfigurasi-lanjutan-control-plane-3},
  style=codestyle,
]
  helm upgrade --install kubernetes-dashboard kubernetes-dashboard/kubernetes-dashboard --create-namespace --namespace kubernetes-dashboard
  echo "setting up user for kubernetes dashboard"
  k3s kubectl apply -f /root/service-account.yaml -f /root/role-binding.yaml

  echo "writing token and starting the dashboard..."
  echo "waiting until all pods in the kubernetes-dashboard namespaces is running"
  k3s kubectl wait pod --all --for=condition=Ready --namespace=kubernetes-dashboard --timeout=-1s
  systemctl start kube-dashboard.service

  echo "done"
\end{lstlisting}

Kode sumber \ref{lst:konfigurasi-lanjutan-control-plane}-\ref{lst:konfigurasi-lanjutan-control-plane-3}
mengkonfigurasi beberapa hal pada \emph{control plane} seperti berikut:

\begin{enumerate}
  
  \item \lstinline{hostname} menentukan nama \emph{hostname} pada \emph{virtual machine}.
    Konfigurasi Cloud-init akan digunakan pada bahasa Golang sehingga \lstinline{%s}
    akan dipakai karena penentuan nama \emph{hostname} dilakukan secara dinamis.

  \item \lstinline{locale} menentukan \emph{locale} yang akan dipakai.

  \item \lstinline{timezone} menentukan waktu yang dipakai yaitu Waktu Indonesia Barat (WIB)

  \item \lstinline{users} menandakan konfigurasi \emph{user} yang akan digunakan. Pada
    contoh tersebut, terdapat dua \emph{user} yang akan dibuat pada \emph{virtual machine},
    yaitu \emph{user default} dan \emph{user} dengan nama user. \emph{User default} merupakan
    \emph{user} admin bawaan.

  \item \lstinline{write_files} berisi \emph{file} apa saja yang akan dibuat beserta konten
    dari \emph{file} tersebut. Pada contoh kode sumber, terdapat 3 \emph{file} yang akan dituliskan.
    \emph{File} pertama adalah \emph{file} berekstensi yaml yang akan digunakan untuk membuat
    ServiceAccount pada klaster Kubernetes. \emph{File} selanjutnya adalah \emph{file} berekstensi
    yaml yang akan digunakan untuk RoleBinding ke ServiceAccount yang dibuat menggunakan
    \emph{file} pertama. \emph{File} terakhir merupakan \emph{file} konfigurasi 
    \emph{service} systemd untuk \emph{dashboard} Kubernetes.

  \item \lstinline{runcmd} merupakan daftar dari \emph{command line} yang akan dijalankan.
    \emph{Command line} yang akan dijalankan adalah \emph{command line} untuk membuat klaster
    Kubernetes menggunakan K3s, menambahkan Repository Helm kubernetes-dashboard ke dalam
    klaster Kubernetes, menggunakan \emph{file} untuk klaster Kubernetes yang sudah dibuat
    melalui \lstinline{write_files}, serta menjalankan \emph{service} systemd untuk
    kubernetes-dashboard.

\end{enumerate}

\begin{lstlisting}[
  style=codestyle,
  caption={Konfigurasi Lanjutan pada \emph{Worker Node} Bagian 1},
  label={lst:konfigurasi-lanjutan-worker-node},
]
#cloud-config
hostname: %s
locale: en_US.UTF-8
timezone: Asia/Jakarta
users:
- default
\end{lstlisting}

\clearpage

\begin{lstlisting}[
  style=codestyle,
  caption={Konfigurasi Lanjutan pada \emph{Worker Node} Bagian 2},
  label={lst:konfigurasi-lanjutan-worker-node-2},
]
- name: user
  groups: sudo
  sudo: ALL=(ALL:ALL) ALL
  plain_text_passwd: user
  lock_passwd: false
  shell: /bin/bash

runcmd:
- echo "installing k3s"
- curl -sfL https://get.k3s.io | INSTALL_K3S_SKIP_DOWNLOAD=true INSTALL_K3S_EXEC="agent --server https://%s:6443 --token %s" sh -s -
- echo "done"
\end{lstlisting}

Kode sumber \ref{lst:konfigurasi-lanjutan-worker-node}-\ref{lst:konfigurasi-lanjutan-worker-node-2}
mengkonfigurasi beberapa hal pada \emph{worker node} seperti berikut:

\begin{enumerate}
  
  \item \lstinline{hostname} menentukan nama \emph{hostname} pada \emph{virtual machine}.
    Konfigurasi Cloud-init akan digunakan pada bahasa Golang sehingga \lstinline{%s}
    akan dipakai karena penentuan nama \emph{hostname} dilakukan secara dinamis.

  \item \lstinline{locale} menentukan \emph{locale} yang akan dipakai.

  \item \lstinline{timezone} menentukan waktu yang dipakai yaitu Waktu Indonesia Barat (WIB)

  \item \lstinline{users} menandakan konfigurasi \emph{user} yang akan digunakan. Pada
    contoh tersebut, terdapat dua \emph{user} yang akan dibuat pada \emph{virtual machine},
    yaitu \emph{user default} dan \emph{user} dengan nama user. \emph{User default} merupakan
    \emph{user} admin bawaan.

  \item \lstinline{runcmd} merupakan daftar dari \emph{command line} yang akan dijalankan.
    \emph{Command line} yang akan dijalankan adalah \emph{command line} untuk bergabung ke
    klaster Kubernetes yang dibuat oleh \emph{control plane} menggunakan K3s.

\end{enumerate}

\subsubsection{Penggunaan \emph{Queue} dan \emph{Worker}}
\label{sec:implementasi-queue-dan-worker}

Mekanisme \emph{queue} akan digunakan pada saat proses \emph{provisioning}
\emph{virtual machine} untuk menghindari permasalahan yang dapat muncul ketika
lebih dari satu \emph{virtual machine} dibuat secara bersamaan. \emph{Queue}
akan berisi permintaan serta parameter yang dibutuhkan untuk membuat
\emph{virtual machine}. Untuk mengerjakan permintaan tersebut, sistem \emph{worker}
akan digunakan.

Redis akan digunakan dalam mengimplementasikan mekanisme \emph{queue}. Redis mendukung
penggunaan basis data dalam bentuk \emph{queue} yang bersifat \emph{First In First Out}.
Pada struktur data \emph{queue}, dua operasi utama yang dibutuhkan adalah penambahan data
ke \emph{queue} dan pengambilan data dari \emph{queue}. Kode sumber penambahan data \emph{queue}
dan pengambilan data dari \emph{queue} dapat dilihat pada kode sumber
\ref{lst:penambahan-data-ke-queue} dan \ref{lst:pengambilan-data-dari-queue}.

\begin{lstlisting}[
  style=codestyle,
  caption={Operasi Penambahan Data ke \emph{Queue}},
  label={lst:penambahan-data-ke-queue},
]
PSEUDOCODE_penambahan data ke queue
FUNC AddToQueue
PARAM CreateInstanceRequest
DO
    CALL redis.leftPush WITH CreateInstanceRequest
DONE
\end{lstlisting}

\clearpage

\begin{lstlisting}[
  style=codestyle,
  caption={Operasi Pengambilan Data dari \emph{Queue}},
  label={lst:pengambilan-data-dari-queue},
]
PSEUDOCODE_pengambilan data dari queue

FUNC PopQueue
RETURN CreateInstanceRequest
DO
    LET queue BE ThisComputerProvisioningQueue
    LET result BE CreateInstanceRequest

    LET data BE CALL redis.rightPop
    SET result TO data

    RETURN result
DONE
\end{lstlisting}

Selain itu, Redis juga dapat digunakan sebagai implementasi dari paradigma \emph{message delivery}
dalam bentuk \emph{publisher} dan \emph{subscriber}. Subjek yang menjadi
\emph{subscribe} dapat menunggu pesan dalam sebuah \emph{channel} dan 
\emph{publisher} dapat mengirim pesan ke dalam \emph{channel} tersebut.
Pada implementasi tugas akhir ini, \emph{message delivery} digunakan
sebagai bentuk komunikasi antara \emph{worker} dengan sistem \emph{provisioning}.
Sistem komunikasi tersebut bertujuan agar \emph{worker} dapat menerima hasil
dari \emph{provisioning}. Kode sumber fungsi \emph{publisher} dan \emph{subscriber}
pada \emph{queue} dapat dilihat pada kode sumber \ref{lst:queue-pub} dan \ref{lst:queue-sub}.

\begin{lstlisting}[
  caption={Operasi \emph{Publisher}},
  label={lst:queue-pub},
  style=codestyle,
]
PSEUDOCODE_publisher

FUNC Publish
PARAM channel, message
DO
    CALL redis.publish WITH channel, message
DONE
\end{lstlisting}

\begin{lstlisting}[
  caption={Operasi \emph{Subscriber}},
  label={lst:queue-sub},
  style=codestyle,
]
PSEUDOCODE_subscriber

FUNC Subscribe
PARAM channel
DO
    CALL redis.subscribe WITH channel
DONE
\end{lstlisting}

\emph{Worker} digunakan untuk menjalankan tugas \emph{provisioning}
yang terdapat pada \emph{queue}. \emph{Worker} akan diimplementasikan
sebagai sebuah goroutine menggunakan bahasa pemrograman Golang. Goroutine merupakan
fitur pada bahasa pemrograman untuk menjalankan sebuah fungsi atau proses
di \emph{background} sehingga tidak terjadi situasi \emph{blocking}. Kode sumber
fungsi \emph{worker} dapat dilihat pada kode sumber \ref{lst:fungsi-worker}.

\clearpage

\begin{lstlisting}[
  caption={Fungsi pada \emph{Worker}},
  label={lst:fungsi-worker},
  style=codestyle,
]
PSEUDOCODE_fungsi worker

FUNC DoWork
DO
    WHILE TRUE
    DO
        LET instanceRequest BE CALL PopQueue
        LET res BE CALL CreateInstance with instanceRequest
        CALL Publish WITH instanceRequest.name, res
    DONE
DONE
\end{lstlisting}

Pada kode sumber \ref{lst:fungsi-worker}, \emph{worker} memanggil fungsi
dari \emph{queue} yaitu fungsi untuk mengambil data dari \emph{queue}
di dalam sebuah \emph{loop} yang tidak akan pernah selesai. Setelah
mendapatkan data permintaan dan parameter yang dibutuhkan untuk \emph{provisioning},
permintaan dan parameter diteruskan ke fungsi untuk membuat \emph{instance}
dari \emph{virtual machine}.

% \subsubsection{Implementasi Websocket}
% \label{sec:implementasi-websocket}
%
% Untuk menunjukkan proses dalam pembuatan \emph{virtual machine}

\subsubsection{Penggunaan QEMU dan Libvirt}
\label{sec:implementasi-libvirt}

Libvirt menyediakan \emph{bindings} untuk banyak bahasa pemrograman
untuk berkomunikasi dengan Libvirt. Libvirt juga menyediakan API untuk
mengelola \emph{virtual machine} dan teknologi virtualisasi.

Dalam implementasi tugas akhir ini, QEMU dan Libvirt akan digunakan.
Untuk menggunakan QEMU sebagai emulator dan interaksi dengan QEMU menggunakan Libvirt,
Libvirt memerlukan konfigurasi koneksi dengan QEMU. Kode sumber untuk konfigurasi
tersebut dapat dilihat pada kode sumber \ref{lst:konfigurasi-qemu-libvirt}.

\begin{lstlisting}[
  caption={Konfigurasi QEMU dengan Libvirt},
  label={lst:konfigurasi-qemu-libvirt},
  style=codestyle,
]
PSEUDOCODE_konfigurasi koneksi dengan Libvirt

FUNC InitLibvirtConnection
RETURN LibvirtConnection
DO
    LET conn BE CALL libvirt.NewConnection WITH "qemu://system"

    RETURN conn
DONE
\end{lstlisting}

Pada kode sumber \ref{lst:konfigurasi-qemu-libvirt}, \lstinline{qemu:///system} akan
digunakan sebagai koneksi ke \emph{daemon} dari Libvirt yang berjalan sebagai \emph{root}.
Koneksi tersebut akan digunakan untuk komunikasi antara \emph{bindings} dari Libvirt yang
dipakai dengan teknologi virtualisasi QEMU.

Proses \emph{provisioning} menggunakan Libvirt terdiri dari beberapa langkah, yaitu:

\begin{enumerate}

  \item Membuat \emph{file} konfigurasi untuk Cloud-init

    Untuk menggunakan Cloud-init dengan konfigurasi yang diinginkan, \emph{file}
    yang berisi konfigurasi Cloud-init harus diubah menjadi \emph{disk} dan digunakan
    pada saat membuat \emph{virtual machine}. Kode sumber untuk membuat \emph{file}
    konfigurasi jaringan terdapat pada kode sumber \ref{lst:pembuatan-file-konfigurasi-jaringan}.

    \begin{lstlisting}[
      caption={Pembuatan \emph{File} Konfigurasi Jaringan},
      label={lst:pembuatan-file-konfigurasi-jaringan},
      style=codestyle,
    ]
PSEUDOCODE_pembuatan file Cloud-init untuk konfigurasi jaringan

FUNC CreateNetwork
DO
    LET networkConfiguration BE CloudNetworkConfiguration

    CALL WriteFile WITH networkConfiguration
DONE
    \end{lstlisting}

    Setelah \emph{file} konfigurasi jaringan telah dibuat, \emph{file}
    konfigurasi lanjutan untuk Cloud-init akan dibuat. Kode sumber untuk membuat
    \emph{file} konfigurasi lanjutan \emph{virtual machine} yang bertugas
    menjadi \emph{control plane} dan \emph{worker node} terdapat pada kode sumber
    \ref{lst:pembuatan-file-konfigurasi-control-plane} dan \ref{lst:pembuatan-file-konfigurasi-worker-node}.

    \begin{lstlisting}[
      caption={Pembuatan \emph{File} Konfigurasi Lanjutan \emph{Control Plane}},
      label={lst:pembuatan-file-konfigurasi-control-plane},
      style=codestyle,
    ]
PSEUDOCODE_pembuatan file Cloud-init untuk Control Plane

FUNC CreateControlPlaneCloudInit
DO
    LET configuration BE ControlPlaneConfiguration
    LET networkConfiguration BE CloudNetworkConfiguration
    
    CALL WriteFile WITH configuration
    CALL cloud-localds WITH configuration, networkConfiguration
DONE
    \end{lstlisting}

    \begin{lstlisting}[
      caption={Pembuatan \emph{File} Konfigurasi Lanjutan \emph{Worker Node}},
      label={lst:pembuatan-file-konfigurasi-worker-node},
      style=codestyle,
    ]
PSEUDOCODE_pembuatan file Cloud-init untuk Worker Node

FUNC CreateWorkerNodeCloudInit
DO
    LET configuration BE WorkerNodeConfiguration
    LET networkConfiguration BE CloudNetworkConfiguration
    
    CALL WriteFile WITH configuration
    CALL cloud-localds WITH configuration, networkConfiguration
DONE
    \end{lstlisting}

    Pada kode sumber \ref{lst:pembuatan-file-konfigurasi-control-plane}
    dan \ref{lst:pembuatan-file-konfigurasi-worker-node}, \emph{disk}
    dibuat menggunakan \emph{tool} \lstinline{cloud-localds}. \emph{Tool}
    tersebut menghasilkan \emph{file} berekstensi iso yang nantinya akan
    dipasangkan ke \emph{virtual machine}.

  \item Menyalin \emph{cloud image}

    \emph{Cloud image} yang sudah dimodifikasi sebelumnya disalin untuk
    setiap \emph{virtual machine} yang akan dibuat dan dilakukan proses
    \emph{resize} sesuai dengan ukuran penyimpanan yang diinginkan oleh
    pengguna. Kode sumber untuk menyalin dan melakukan proses \emph{resize}
    terdapat pada kode sumber \ref{lst:penyalinan-cloud-image}

    \clearpage

    \begin{lstlisting}[
      caption={Penyalinan \emph{cloud image}},
      label={lst:penyalinan-cloud-image},
      style=codestyle,
    ]
PSEUDOCODE_penyalinan cloud image

FUNC CopyImage
PARAM instanceName, instanceConfig
DO
    LET data BE CALL ReadFile WITH baseImage
    LET destinationPath BE image_dir + "/" + instanceName

    CALL WriteFile WITH destinationPath, data
    CALL ResizeImage WITH instanceConfig.Storage
DONE
    \end{lstlisting}

  \item Membuat \emph{virtual machine}

    Untuk membuat \emph{virtual machine} melalui Libvirt, konfigurasi
    mengenai \emph{virtual machine} tersebut perlu diberikan. Libvirt menggunakan
    \emph{file} berekstensi xml untuk konfigurasi \emph{virtual machine}.
    Kode sumber untuk mengatur konfigurasi xml \emph{virtual machine} terdapat
    pada kode sumber \ref{lst:konfigurasi-xml-virtual-machine}

    \begin{lstlisting}[
      caption={Konfigurasi xml \emph{Virtual Machine}},
      label={lst:konfigurasi-xml-virtual-machine},
      style=codestyle,
    ]
PSEUDOCODE_pengaturan konfigurasi xml virtual machine

FUNC CreateVirtualMachineXml
PARAM instanceName, instanceConfig
RETURN string
DO
    LET instanceIso BE image_dir + "/" + instanceName
    LET cloudConfig BE cloud_config_dir + "/" + instanceName
    LET xmlConfig BE libvirtXml WITH instanceIso, cloudConfig

    RETURN xmlConfig
DONE
    \end{lstlisting}

    Setelah konfigurasi xml dari \emph{virtual machine} telah dibuat, Libvirt
    dapat menggunakan konfigurasi tersebut untuk membuat \emph{virtual machine}.
    Kode sumber untuk membuat \emph{virtual machine} menggunakan konfigurasi
    xml dapat dilihat pada kode sumber \ref{lst:pembuatan-vm}.

    \begin{lstlisting}[
      caption={Pembuatan \emph{Virtual Machine}},
      label={lst:pembuatan-vm},
      style=codestyle,
    ]
PSEUDOCODE_pembuatan virtual machine

FUNC CreateVirtualMachine
PARAM instanceRequest
RETURN 
DO
    LET instanceName BE random_string
    LET conn BE CALL InitLibvirtConnection

    CALL CreateNetwork
    LET configuration BE CALL CreateVMCloudInit
    CALL CopyImage WITH instanceName, instanceRequest
    CALL conn.CreateVM WITH configuration
DONE
    \end{lstlisting}

  \item Menunggu proses Cloud-init

    Cloud-init memerlukan beberapa waktu untuk menyelesaikan tugasnya.
    Beberapa proses memerlukan proses yang dijalankan oleh Cloud-init
    untuk selesai terlebih dahulu. Untuk menunggu proses Cloud-init selesai,
    ditunjukkan pada kode sumber \ref{lst:menunggu-cloud-init}.

    \begin{lstlisting}[
      caption={Menunggu Proses Cloud-init Selesai},
      label={lst:menunggu-cloud-init},
      style=codestyle,
    ]
PSEUDOCODE_menunggu proses Cloud-init selesai

FUNC WaitCloudInit
PARAM instanceRequest
DO
    LET dom BE VMDomain
    LET waitCloudInit BE "cloud-init status --wait"
    CALL dom.QemuAgentCommand WITH waitCloudInit
DONE
    \end{lstlisting}

\end{enumerate}

\section{Implementasi Situs Web}
\label{sec:implementas-situs-web}

Situs web yang dibuat akan digunakan sebagai pengguna untuk membuat
klaster Kubernetes menggunakan \emph{virtual machine} yang dibuat
oleh sistem \emph{provisioning}. Implementasi situs web dibagi menjadi
dua yaitu implementasi bagian \emph{frontend} dan bagian \emph{backend}.

\subsection{Implementasi \emph{Frontend}}
\label{subsec:implementas-frontend}

Situs web \emph{dashboard} merupakan antarmuka yang akan digunakan oleh \emph{user}
untuk membuat klaster Kubernetes. Pada situs web tersebut, terdapat
informasi mengenai identitas komputer yang siap
menerima permintaan untuk membuat \emph{virtual machine}.
Tampilan dari \emph{dashboard} dapat dilihat pada gambar
dan \ref{fig:dashboard-with-node}.

\begin{figure}[H]
  \centering
  \fbox{\includegraphics[scale=0.3]{gambar/dashboard-with-node.png}}
  \caption{Tampilan \emph{Dashboard}}
  \label{fig:dashboard-with-node}
\end{figure}

Pada contoh gambar \ref{fig:dashboard-with-node}, terdapat satu komputer \emph{worker}
yang siap untuk menerima permintaan, yaitu komputer yang memiliki \emph{hostname} hudson dan alamat
ip 10.125.155.249.

% TODO: implementasi backend
\subsection{Implementasi \emph{Backend}}
\label{subsec:implementas-backend}

Bagian \emph{backend} dari situs web bertanggung jawab untuk menangani permintaan
yang dibuat pengguna pada bagian \emph{frontend}. Selain itu, bagian \emph{backend}
juga bertanggung jawab untuk melakukan komunikasi dengan komputer \emph{worker} menggunakan
protokol RPC melalui gRPC. Garis besar pertukaran informasi dan komunikasi
dapat dilihat pada gambar \ref{fig:frontend-backend-worker}.

\begin{figure}[H]
  \centering
  \fbox{\includegraphics[scale=0.55]{gambar/frontend-backend-worker.png}}
  \caption{Komunikasi \emph{Frontend-Backend-Worker}}
  \label{fig:frontend-backend-worker}
\end{figure}

\cleardoublepage

% Bab 4 pengujian dan analisis
\chapter{HASIL DAN PEMBAHASAN}
\label{chap:hasil-pembahasan}

Pada bab ini akan dijelaskan mengenai hasil implementasi
dan pengujian dari sistem \emph{provisioning} yang telah
dibuat sebelumnya pada bab 3. Pengujian yang dilakukan
menggunakan laptop penulis dan komputer dari laboratorium Rekayasa Perangkat
Lunak Teknik Informatika ITS. Spesifikasi dari laptop dan komputer
yang digunakan dapat dilihat pada tabel \ref{tb:spesifikasi-laptop-pengujian},
\ref{tb:spesifikasi-komputer-pengujian}, dan \ref{tb:spesifikasi-komputer-pengujian-2}.

\begin{longtable}{|c|c|}
  \caption{Spesifikasi Laptop untuk Pengujian}
  \label{tb:spesifikasi-laptop-pengujian} \\
  \hline
  OS     & Fedora Linux 42.20250614.0 (Kinoite) x86\_64 \\
  \hline
  Kernel & Linux 6.14.9-300.fc42.x86\_64                \\
  \hline
  CPU    & Intel(R) Core(TM) i7-10750H (12) @ 5.00 GHz       \\
  \hline
  Integrated GPU   & Intel UHD Graphics @ 1.15 GHz [Integrated]       \\
  \hline
  Discrete GPU    & NVIDIA GeForce GTX 1650 Ti Mobile [Discrete]       \\
  \hline
  RAM    & 15848MiB       \\
  \hline
\end{longtable}

\begin{longtable}{|c|c|}
  \caption{Spesifikasi Komputer Jenis Satu untuk Pengujian}
  \label{tb:spesifikasi-komputer-pengujian} \\
  \hline
  OS     & Ubuntu 24.04.2 LTS x86\_64 \\
  \hline
  Kernel & 6.11.0-26-generic          \\
  \hline
  CPU    & 12th Gen Intel i7-12700 (20) @ 4.800GHz       \\
  \hline
  GPU    & Intel AlderLake-S GT1       \\
  \hline
  RAM    & 31834MiB       \\
  \hline
\end{longtable}

\begin{longtable}{|c|c|}
  \caption{Spesifikasi Komputer Jenis Dua untuk Pengujian}
  \label{tb:spesifikasi-komputer-pengujian-2} \\
  \hline
  OS     & Ubuntu 24.04.2 LTS x86\_64 \\
  \hline
  Kernel & 6.11.0-24-generic          \\
  \hline
  CPU    & 12th Gen Intel i9-12900k (24) @ 5.100GHz       \\
  \hline
  GPU    & NVIDIA GeForce RTX 3080 Ti       \\
  \hline
  RAM    & 63998MiB       \\
  \hline
\end{longtable}

\section{Hasil Implementasi}
\label{sec:hasil-implementasi}

Subbab ini menjelaskan hasil implementasi yang dilakukan pada
Tugas Akhir.

\subsection{Implementasi Linux Bridge}
\label{subsec:implementasi-linux-bridge}

Linux \emph{bridge} yang telah dijelaskan sebelumnya dibuat
agar \emph{virtual machine} memiliki jangkauan alamat IP yang sama
dengan jangkauan alamat IP dari komputer \emph{host}. Jangkauan IP yang
sama dengan komputer \emph{host} diperlukan agar \emph{virtual machine}
yang berada di komputer A dapat berkomunikasi dengan \emph{virtual machine}
yang berada di komputer B.

Pada kode sumber \ref{cli:host-network-interface}, komputer \emph{host} memiliki
\emph{network interface} untuk ethernet bernama enp4s0. Linux \emph{bridge} yang
dibuat untuk implementasi tugas akhir ini adalah k3s-br0 dan enp4s0 menjadi
\emph{slave} dari Linux \emph{bridge} k3s-br0. Alamat IP dan jangkauan IP dari
komputer \emph{host} tersebut adalah 10.21.73.17 dengan \emph{subnet mask} /24
yang berarti dalam \emph{local area network} tersebut terdapat 255 alamat
IP yang dapat dibagikan oleh DHCP server.

{\renewcommand{\lstlistingname}{Instruksi Terminal}
\begin{lstlisting}[
  style=clistyle,
  caption={Informasi \emph{Network Interface} pada Komputer \emph{Host}},
  label={cli:host-network-interface}
]
...
2: enp4s0: <BROADCAST,MULTICAST,UP,LOWER_UP> mtu 1500 qdisc mq master k3s-br0 state UP group default qlen 1000
    link/ether c8:7f:54:6c:47:ff brd ff:ff:ff:ff:ff:ff
...
8: k3s-br0: <BROADCAST,MULTICAST,UP,LOWER_UP> mtu 1500 qdisc noqueue state UP group default qlen 1000
    link/ether 92:e9:0d:63:4e:32 brd ff:ff:ff:ff:ff:ff
    inet 10.21.73.107/24 brd 10.21.73.255 scope global dynamic noprefixroute k3s-br0
       valid_lft 94877sec preferred_lft 94877sec
    inet6 fe80::1d88:4787:24c3:2a99/64 scope link noprefixroute
       valid_lft forever preferred_lft forever
\end{lstlisting}
}

Agar \emph{virtual machine} memiliki jangkauan alamat IP yang sama dengan 
jangkauan alamat IP address yang didapat oleh komputer \emph{host}, \emph{virtual machine}
dapat menggunakan Linux \emph{bridge} yang sudah dibuat sebagai sumber jaringan
yang dipakai. Konfigurasi tersebut dapat dilakukan melalui konfigurasi berbentuk
xml seperti yang dapat dilihat pada kode sumber \ref{code:vm-interface-configuration}.
Alamat IP yang didapat pada \emph{virtual machine} dapat dilihat pada kode
sumber \ref{cli:vm-network-interface}.

\begin{lstlisting}[
  language=xml,
  style=codestyle,
  caption={Konfigurasi \emph{Interface} Jaringan pada \emph{Virtual Machine}},
  label={code:vm-interface-configuration}
]
...
<interface type='bridge'>
  <mac address='52:54:00:cf:ee:9e'/>
  <source bridge='k3s-br0'/>
  <model type='virtio'/>
  <alias name='net0'/>
  <address type='pci' domain='0x0000' bus='0x01' slot='0x00' function='0x0'/>
</interface>
...
\end{lstlisting}

{\renewcommand{\lstlistingname}{Instruksi Terminal}
\begin{lstlisting}[
  style=clistyle,
  caption={Informasi \emph{Network Interface} pada \emph{Virtual Machine}},
  label={cli:vm-network-interface}
]
...
2: enp1s0: <BROADCAST,MULTICAST,UP,LOWER_UP> mtu 1500 qdisc fq_codel state UP group default qlen 1000
    link/ether 52:54:00:cf:ee:9e brd ff:ff:ff:ff:ff:ff
    inet 10.21.73.134/24 brd 10.21.73.255 scope global dynamic noprefixroute enp1s0
       valid_lft 105859sec preferred_lft 105859sec
    inet6 fe80::5054:ff:fecf:ee9e/64 scope link proto kernel_ll
       valid_lft forever preferred_lft forever
...
\end{lstlisting}
}

Pada kode sumber \ref{cli:vm-network-interface}, alamat IP yang didapat oleh
\emph{virtual machine} adalah 10.21.73.134 dan memiliki \emph{subnet mask} /24.
Alamat IP dan \emph{subnet mask} yang didapat oleh \emph{virtual machine} sama
dengan alamat IP dan \emph{subnet mask} yang didapat oleh komputer \emph{host}.
Informasi lanjutan mengenai Linux \emph{bridge} dan \emph{network interfaces}
yang tersambung dapat dilihat pada kode sumber \ref{cli:bridge-interfaces}.

{\renewcommand{\lstlistingname}{Instruksi Terminal}
\begin{lstlisting}[
  style=clistyle,
  caption={Linux \emph{Bridge} dan \emph{Network Interfaces} yang Tersambung},
  label={cli:bridge-interfaces}
]
bridge name     bridge id               STP enabled     interfaces
k3s-br0         8000.92e90d634e32       yes             enp4s0
                                                        vnet7
\end{lstlisting}
}

Pada kode sumber \ref{cli:bridge-interfaces}, \emph{network interfaces} yang
berada di bawah Linux \emph{bridge} k3s-br0 adalah \emph{network interface}
enp4s0 dan vnet7. \emph{Network interface} enp4s0 merupakan \emph{network interface}
untuk ethernet sedangkan vnet7 merupakan \emph{virtual network} yang dibuat
dan tersambung dengan \emph{network interface} dari \emph{virtual machine}.
Gambaran mengenai hal tersebut dapat dilihat pada \ref{cli:vnet-big}.

\begin{lstlisting}[
  style=clistyle,
  caption={Hubungan Antar \emph{Devices}},
  label={cli:vnet-big}
]
VM's ethernet interface <-> vnet <-> Linux bridge <-> physical NIC
\end{lstlisting}

\subsection{Implementasi Cloud-init}
\label{subsec:implementasi-cloud-init}

Setiap \emph{virtual machine} yang dibuat memiliki sebuah \emph{file} berekstensi iso khusus
yang berisikan \emph{file} konfigurasi Cloud-init. \emph{File} tersebut kemudian akan digunakan
saat mendefinisikan konfigurasi xml dari \emph{virtual machine}. Contoh definisi konfigurasi
xml tersebut dapat dilihat pada kode sumber \ref{lst:xml-cloud-init}.

\lstinputlisting[
  language=go,
  style=codestyle,
  emphstyle=\color{black}\bfseries\underbar,
  emph={seedFile},
  caption={Konfigurasi xml dengan \emph{File} Cloud-init},
  label={lst:xml-cloud-init}
]{program/xml-cloud-init-iso.go}

Cloud-init yang berjalan pada \emph{virtual machine} menghasilkan \emph{file}
\emph{log} dari proses konfigurasi Cloud-init. \emph{File} tersebut dapat digunakan
untuk proses \emph{debug} konfigurasi yang digunakan. Cloud-init menghasilkan
dua \emph{file logging}, yaitu \emph{log} dari proses Cloud-init itu sendiri
dan \emph{log} dari \emph{command} yang dijalankan oleh Cloud-init. Hasil
dari dua \emph{file} tersebut dapat dilihat pada gambar \ref{fig:log-cloud-init}
dan gambar \ref{fig:log-output-cloud-init}.

\begin{figure}[H]
  \centering
  \fbox{\includegraphics[scale=0.3]{gambar/cloud-init.png}}
  \caption{\emph{Log} Cloud-init}
  \label{fig:log-cloud-init}
\end{figure}

\begin{figure}[H]
  \centering
  \fbox{\includegraphics[scale=0.3]{gambar/cloud-init-output.png}}
  \caption{\emph{Log Command} yang Dijalankan Cloud-init}
  \label{fig:log-output-cloud-init}
\end{figure}

\subsection{\emph{Error Handling}}
\label{subsec:error-handling}

Proses \emph{provisioning virtual machine} untuk pembuatan
klaster Kubernetes tidak selalu berjalan baik. Terdapat banyak
faktor yang dapat membuat proses \emph{provisioning} gagal. Untuk
mengatasi hal tersebut, proses \emph{provisioning} pada implementasi
tugas akhir ini memiliki sifat atomik, yaitu jika terjadi \emph{error}
saat membuat \emph{virtual machine} dan \emph{virtual machine} sudah terbuat,
\emph{virtual machine} tersebut akan dihapus.

\lstinputlisting[
  language=go,
  style=codestyle,
  caption={Kode Sumber Penghapusan \emph{Virtual Machine}},
  label={lst:vm-delete-function}
]{program/delete-vm.go}

\clearpage

\lstinputlisting[
  language=go,
  style=codestyle,
  emphstyle=\color{black}\bfseries\underbar,
  emph={deleteInstance},
  caption={Contoh Penggunaan \emph{Error Handling}},
  label={lst:error-handling-example}
]{program/error-handling.go}

Pada kode sumber \ref{lst:vm-delete-function}, fungsi \lstinline{deleteInstance}
akan mematikan \emph{virtual machine} dan menghapus \emph{file} yang berkaitan
dengan \emph{virtual machine} tersebut seperti \emph{cloud image} yang dipakai.
Bentuk implementasi dari \emph{error handling} tersebut dapat dilihat pada kode
sumber \ref{lst:error-handling-example}. Pada kode sumber tersebut, jika terjadi
\emph{error} pada saat membuat \emph{virtual machine}, maka \emph{virtual machine}
tersebut akan dihapus beserta semua \emph{file} yang berkaitan.

\subsection{Mekanisme \emph{Mutex}}

Pada saat proses pembuatan VM, VM memerlukan sumber daya seperti \emph{storage}. Pada
implementasi Tugas Akhir ini, \emph{storage} yang digunakan merupakan \emph{image} file yang
akan diubah ukurannya menjadi lebih besar sesuai dengan jumlah \emph{storage} yang sudah
ditetapkan untuk setiap grup. Jika terjadi proses perubahan ukuran dari banyak
VM dalam satu waktu akan memungkinkan untuk terjadinya masalah ketika banyak VM tersebut
mencoba untuk menambah ukuran \emph{storage} dalam satu waktu yang sama.

Selain itu, VM memiliki sistem operasi tersendiri. Sumber daya yang digunakan
untuk menjalankan sistem operasi tersebut.

\section{Hasil Pengujian}
\label{sec:hasil-pengujian}

Subbab ini menjelaskan pengujian yang dilakukan pada
implementasi Tugas Akhir.

\subsection{Pengujian Pembuatan \emph{Virtual Cluster} Lingkungan Lokal}
\label{subsec:pengujian-pembuatan-vc}

Pengujian proses \emph{provisioning} pada komputer lokal bertujuan untuk
menguji apakah sistem yang telah dibangun dapat membuat \emph{virtual cluster}
yang sesuai dengan kriteria \emph{user}. Pengujian pada lingkungan lokal ini
tidak memerlukan Linux \emph{bridge} karena jaringan internet yang digunakan
oleh \emph{virtual machine} akan menggunakan \emph{Network Address Translation} (NAT)
dari \emph{host}, sehingga setiap \emph{virtual machine} dapat berkomunikasi satu sama
lain tanpa menggunakan Linux \emph{bridge}.

Skenario pengujian yang akan dilakukan pada lingkungan lokal adalah
membuat klaster Kubernetes dengan satu \emph{virtual machine} dan dua \emph{virtual machine}.
Pada klaster Kubernetes yang terdiri satu \emph{virtual machine}, \emph{virtual machine}
tersebut bertugas sebagai \emph{control plane}. Sedangkan pada klaster Kubernetes yang terdiri
dari dua \emph{virtual machine}, satu dari \emph{virtual machine} tersebut bertugas sebagai
\emph{control plane} dan sisanya sebagai \emph{worker node}.

Setelah pembuatan klaster selesai, \emph{dashboard} akan menampilkan
token untuk dapat mengakses \emph{dashboard} Kubernetes dari klaster
yang dibuat. Selain itu, tombol untuk mengakses klaster juga akan ditampilkan
dan dapat ditekan oleh \emph{user} untuk menuju situs web \emph{dashboard}
klaster Kubernetes.

\begin{figure}[H]
  \centering
  \fbox{\includegraphics[scale=0.3]{gambar/website-create-process.png}}
  \caption{Proses Pembuatan Klaster}
  \label{fig:proses-pembuatan-klaster}
\end{figure}

\begin{figure}[H]
  \centering
  \fbox{\includegraphics[scale=0.3]{gambar/website-create-process-done-local.png}}
  \caption{Proses Pembuatan Klaster Selesai}
  \label{fig:proses-pembuatan-klaster-selesai}
\end{figure}

\begin{figure}[H]
  \centering
  \includegraphics[scale=0.3]{gambar/worker-create-cluster-process-local.png}
  \caption{\emph{Log} pada Komputer \emph{Worker}}
  \label{fig:worker-create-cluster-process-local}
\end{figure}

\begin{figure}[H]
  \centering
  \includegraphics[scale=0.3]{gambar/kubernetes-dashboard-access-local-with-nodes.png}
  \caption{Daftar \emph{Nodes} pada Klaster dengan Satu \emph{Virtual Machine}}
  \label{fig:daftar-nodes-pada-dashboard-kubernetes}
\end{figure}

Berdasarkan gambar-gambar di atas, sistem \emph{provisioning} dapat membuat
\emph{virtual machine} yang secara otomatis tergabung dalam sebuah klaster Kubernetes.
Selain itu, \emph{control plane} juga menyediakan \emph{dashboard} Kubernetes yang dapat
digunakan oleh pengguna untuk berinteraksi dengan klaster Kubernetes tersebut.

\subsection{Pengujian Pembuatan \emph{Virtual Cluster} Lingkungan \emph{Production}}
\label{subsec:pengujian-pembuatan-vc-prod}

Pada lingkungan \emph{production}, \emph{virtual machine} yang tergabung dalam
satu klaster tidak selalu berada dalam satu komputer fisik \emph{worker} yang sama.
Pengujian dilakukan dengan cara membuat klaster berisi dua atau lebih \emph{virtual machine}
yang terdiri dari satu \emph{control plane} dan sisanya sebagai \emph{worker node}.
Semua \emph{Virtual machine} tersebut tidak selalu berada di komputer \emph{worker}
yang sama.

Pada subbab pengujian ini, akan dibuat sebuah klaster yang terdiri dari dua
\emph{virtual machine} yang berada di dua komputer fisik yang berbeda. Pada gambar
\ref{fig:nodes-2-komputer-berbeda-1}, klaster tersebut memiliki \emph{nodes} yang
bernama eovugekt dan gxlshrqm. Komputer fisik dari dua \emph{virtual machine}
tersebut dapat dilihat pada gambar \ref{fig:vm-komputer-fisik-1}.

Gambar \ref{fig:vm-komputer-fisik-1} menunjukkan bahwa \emph{nodes} gxlshrqm berada
pada komputer fisik rpl-1 dan \emph{nodes} eovugekt berada pada
komputer fisik rpl-02. Untuk pengujian \emph{multi-tenancy}, akan
dibuat klaster Kubernetes lagi dari dua komputer fisik tersebut.

Pada gambar \ref{fig:nodes-2-komputer-berbeda-2}, klaster yang baru
memiliki \emph{nodes} bernama qxwuyusc dan vsbftqms. Gambar \ref{fig:vm-komputer-fisik-2}
menunjukkan bahwa \emph{node} qxwuyusc berada pada komputer fisik rpl-02 yang juga
merupakan komputer fisik dari \emph{node} eovugekt. Dari hal tersebut, dapat
dilihat bahwa komputer fisik rpl-02 digunakan oleh dua klaster dan dua pengguna
yang berbeda.

\begin{figure}[H]
  \centering
  \includegraphics[scale=0.3]{gambar/two-nodes-difference-computer-dashboard.png}
  \caption{Daftar \emph{Nodes} pada Klaster 1}
  \label{fig:nodes-2-komputer-berbeda-1}
\end{figure}

\begin{figure}[H]
  \centering
  \includegraphics[scale=0.3]{gambar/ssh-nodes-list-1.png}
  \caption{Daftar \emph{Virtual Machine} Setelah Pembuatan Klaster 1}
  \label{fig:vm-komputer-fisik-1}
\end{figure}

\begin{figure}[H]
  \centering
  \includegraphics[scale=0.3]{gambar/two-nodes-difference-computer-dashboard-2.png}
  \caption{Daftar \emph{Nodes} pada Klaster 2}
  \label{fig:nodes-2-komputer-berbeda-2}
\end{figure}

\begin{figure}[H]
  \centering
  \includegraphics[scale=0.3]{gambar/ssh-nodes-list-2.png}
  \caption{Daftar \emph{Virtual Machine} Setelah Pembuatan Klaster 2}
  \label{fig:vm-komputer-fisik-2}
\end{figure}

\subsection{Pengujian \emph{Deployment} pada Klaster Kubernetes}
\label{subsec:pengujian-deployment}

Pengujian berupa \emph{Deployment} pada Klaster Kubernetes dilakukan dengan
menggunakan Nginx yang akan diekspos. Kemudian memeriksa apakah \emph{Deployment}
tersebut dapat diakses atau tidak. \emph{Command line} yang digunakan untuk
membuat \emph{Deployment} Nginx kemudian mengekspos agar dapat diakses
dapat dilihat pada \ref{cli:cli-deployment-nginx}.

{\renewcommand{\lstlistingname}{Instruksi Terminal}
\begin{lstlisting}[
  style=clistyle,
  caption={\emph{Command Line} untuk Membuat \emph{Deployment} Nginx},
  label={cli:cli-deployment-nginx},
]
$ kubectl create deployment nginx --image=nginx
deployment.apps/nginx created

$ kubectl expose deployment nginx --type=NodePort --port=80 --target-port=80
service/nginx exposed

$ kubectl get svc nginx
NAME    TYPE       CLUSTER-IP   EXTERNAL-IP   PORT(S)        AGE
nginx   NodePort   10.43.5.58   <none>        80:32673/TCP   14s
\end{lstlisting}
}

Berdasarkan instruksi terminal \ref{cli:cli-deployment-nginx}, \emph{Deployment}
Nginx pada klaster tersebut diekspos ke \emph{port} 32673. Hasil pengaksesan
\emph{Deployment} tersebut dapat dilihat pada gambar \ref{fig:nginx-deployment}.
Dapat dilihat juga pada gambar \ref{fig:nginx-deployment} bahwa \emph{port} yang diakses adalah
\emph{port} 32673.

\begin{figure}[H]
  \centering
  \includegraphics[scale=0.3]{gambar/nginx-deployment.png}
  \caption{Halaman Situs Web Nginx \emph{Deployment} pada Klaster}
  \label{fig:nginx-deployment}
\end{figure}

\subsection{Pengujian Waktu Pembuatan Klaster Kubernetes}
\label{subsec:pengujian-deployment}

Pengujian ini dilakukan untuk mengetahui waktu yang diperlukan untuk membuat
klaster Kubernetes dengan beberapa variabel yang berbeda seperti berikut:

\begin{itemize}

  \item[-] 1 \emph{worker}
  
  \item[-] 2 \emph{worker}

  \item[-] 3 \emph{worker}

\end{itemize}

Untuk semua percobaan dengan variabel yang sudah disebutkan sebelumnya, jumlah \emph{virtual cluster}
yang dibuat adalah 5 dengan masing-masing \emph{virtual cluster} terdiri dari 2 \emph{node}
dan tiap \emph{node} memiliki 2 VCPU, 2 GB \emph{memory}, dan 4 GB \emph{storage}. Ketentuan
sumber daya tersebut berdasarkan sumber daya minimum yang dibutuhkan oleh K3s dalam menjalankan
klaster, yaitu \emph{control plane} dengan 2 \emph{cores} CPU dan 2GB RAM, dan \emph{node}
dengan 1 \emph{core} CPU dan 512MB RAM. K3s tidak menyebutkan ukuran penyimpanan minimum yang dibutuhkan
untuk menjalankan klaster, sehingga penentuan ukuran penyimpanan berdasarkan pada ukuran penyimpanan
minimum untuk Ubuntu \emph{cloud image}, yaitu 4GB. Selain itu, ukuran penyimpanan yang besar akan
membuat proses \emph{provisioning} VM membutuhkan waktu yang lebih lama, terlebih lagi jika perangkat
keras untuk penyimpanan tidak menggunakan \emph{Solid State Drive} (SSD) dan masih menggunakan
\emph{Hard Disk Drive} (HDD). Tabel waktu yang diperlukan untuk menyelesaikan \emph{job}
pembuatan 5 \emph{virtual cluster} dapat dilihat pada tabel \ref{tb:test-1-worker}, \ref{tb:test-2-worker},
dan \ref{tb:test-3-worker}.

\begin{longtable}{|c|c|}
  \caption{Waktu yang Dibutuhkan Menggunakan 1 \emph{Worker}}
  \label{tb:test-1-worker}                                   \\
  \hline
  \rowcolor[HTML]{C0C0C0}
  \textbf{Percobaan} & \textbf{Waktu Total yang Dibutuhkan} \\
  \hline
  1            & 14 menit 40 detik \\
  2            & 11 menit  0 detik \\
  3            & 12 menit 30 detik \\
  4            & 15 menit  5 detik \\
  5            & 14 menit 15 detik \\
  \hline
  \textbf{Rata-Rata} & \textbf{13 Menit 30 detik} \\
  \hline
\end{longtable}

\begin{longtable}{|c|c|}
  \caption{Waktu yang Dibutuhkan Menggunakan 2 \emph{Worker}}
  \label{tb:test-2-worker}                                   \\
  \hline
  \rowcolor[HTML]{C0C0C0}
  \textbf{Percobaan} & \textbf{Waktu Total yang Dibutuhkan} \\
  \hline
  1 &  9 menit  5 detik \\
  2 &  9 menit 50 detik \\
  3 & 13 menit  0 detik \\
  4 &  9 menit 35 detik \\
  5 & 11 menit 50 detik \\
  \hline
  \textbf{Rata-Rata} & \textbf{10 Menit 40 detik} \\
  \hline
\end{longtable}

\begin{longtable}{|c|c|}
  \caption{Waktu yang Dibutuhkan Menggunakan 3 \emph{Worker}}
  \label{tb:test-3-worker}                                   \\
  \hline
  \rowcolor[HTML]{C0C0C0}
  \textbf{Percobaan} & \textbf{Waktu Total yang Dibutuhkan} \\
  \hline
  1 & 11 menit 50 detik \\
  2 &  9 menit 15 detik \\
  3 &  8 menit 45 detik \\
  4 &  9 menit 30 detik \\
  5 &  9 menit 45 detik \\
  \hline
  \textbf{Rata-Rata} & \textbf{9 Menit 49 detik} \\
  \hline
\end{longtable}

Dalam proses pengujian, 5 permintaan pembuatan \emph{virtual cluster}
tersebut terjadi dalam satu waktu. Semua permintaan pembuatan tersebut
akan dimasukkan ke dalam \emph{queue} yang kemudian akan dikerjakan oleh
\emph{worker}.

Pada saat \emph{worker} mengerjakan permintaan pembuatan \emph{virtual cluster},
\emph{worker} akan mengirim permintaan tersebut ke \emph{worker node}
secara \emph{random}. \emph{List} dari \emph{worker node} yang digunakan
dalam proses pengujian ini dapat dilihat pada gambar \ref{fig:testing-worker-nodes}.
Berdasarkan tabel waktu total yang dibutuhkan untuk menyelesaikan
semua permintaan, rata-rata waktu yang dibutuhkan untuk menyelesaikan semua permintaan
tersebut menjadi lebih cepat ketika \emph{worker} yang digunakan meningkat.
Namun, total waktu yang dibutuhkan untuk menyelesaikan semua permintaan
tersebut tidak selalu lebih cepat ketika menggunakan \emph{worker} yang lebih banyak.
Salah satu contohnya adalah percobaan 3 dengan 2 \emph{worker}. Waktu yang
dibutuhkan adalah 13 menit 0 detik. Pada beberapa percobaan dengan menggunakan
hanya 1 \emph{worker}, percobaan 2 menghasilkan waktu 11 menit 0 detik. Hal tersebut
dapat terjadi jika \emph{worker node} yang sama dipilih pada saat percobaan 3 dengan 2 \emph{worker}

\begin{figure}[H]
  \centering
  \includegraphics[scale=0.365]{gambar/testing-workers.png}
  \caption{\emph{Worker Node} yang Digunakan untuk Pengujian.}
  \label{fig:testing-worker-nodes}
\end{figure}

% Contoh pembuatan tabel
% \begin{longtable}{|c|c|c|}
%   \caption{Hasil Pengukuran Energi dan Kecepatan}
%   \label{tb:EnergiKecepatan}                                   \\
%   \hline
%   \rowcolor[HTML]{C0C0C0}
%   \textbf{Energi} & \textbf{Jarak Tempuh} & \textbf{Kecepatan} \\
%   \hline
%   10 J            & 1000 M                & 200 M/s            \\
%   20 J            & 2000 M                & 400 M/s            \\
%   30 J            & 4000 M                & 800 M/s            \\
%   40 J            & 8000 M                & 1600 M/s           \\
%   \hline
% \end{longtable}

\cleardoublepage

% Bab 5 penutup
\chapter{KESIMPULAN DAN SARAN}
\label{chap:penutup}

Pada bab ini akan dipaparkan kesimpulan dari hasil implementasi
yang sudah dilakukan. Selain itu, saran mengenai hal yang dapat
dilakukan untuk mengembangkan implementasi ini juga akan dipaparkan.

\section{Kesimpulan}
\label{sec:kesimpulan}

Berdasarkan hasil implementasi yang dilakukan, berikut adalah
kesimpulan yang dapat diambil berdasarkan permasalahan yang diangkat
pada tugas akhir ini:

\begin{enumerate}[nolistsep]

  \item Pembuatan sistem \emph{provisioning} untuk menerapkan \emph{multi-tenancy}
    dalam klaster Kubernetes dapat dilakukan dengan cara mengaplikasikan \emph{tools}
    untuk membuat \emph{virtual machine} seperti Libvirt, Cloud-init, dan \emph{cloud images}.

  \item Penerapan \emph{multi-tenancy} pada implementasi ini adalah
    komputer fisik yang digunakan sebagai tempat \emph{provisioning} dapat
    membuat lebih dari satu \emph{virtual machine} yang tergabung dalam
    klaster yang berbeda. Implementasi tersebut menyebabkan komputer
    fisik tersebut melayani lebih dari satu \emph{user} sehingga
    komputer fisik tersebut memenuhi kriteria \emph{multi-tenancy}.

  \item Pembuatan \emph{virtual machine} yang secara otomatis tergabung
    dalam sebuah klaster Kubernetes dapat dicapai melalui beberapa \emph{tools}
    seperti K3s, Libvirt dan Cloud-init. K3s digunakan sebagai distribusi
    Kubernetes yang berukuran lebih kecil untuk membuat klaster Kubernetes,
    Libvirt digunakan untuk berinteraksi dengan hypervisor agar dapat membuat 
    \emph{virtual machine} dan Cloud-init digunakan untuk mengotomatisasi proses
    pembuatan dan penggabungan klaster Kubernetes.

\end{enumerate}

\section{Saran}
\label{chap:saran}

Untuk pengembangan lebih lanjut pada sistem \emph{provisioning}
untuk membuat klaster Kubernetes adalah sebagai berikut:

\begin{enumerate}[nolistsep]

  \item Meningkatkan kualitas dari antarmuka pengguna dan pengalaman
    pengguna pada situs web \emph{dashboard} sebagai antarmuka
    yang digunakan oleh \emph{user} untuk membuat klaster.

  \item Meningkatkan keamanan dengan cara mengisolasi \emph{virtual machine}
    agar tidak dapat diakses oleh \emph{virtual machine} lain yang tidak
    dalam satu klaster.

\end{enumerate}

\cleardoublepage

\chapter*{DAFTAR PUSTAKA}
\addcontentsline{toc}{chapter}{DAFTAR PUSTAKA}
\renewcommand\refname{}
\vspace{2ex}
\renewcommand{\bibname}{}
\begingroup
\def\chapter*#1{}
\printbibliography
\endgroup
\cleardoublepage

% Biografi penulis
\chapter*{BIOGRAFI PENULIS}

\addcontentsline{toc}{chapter}{BIOGRAFI PENULIS}

\vspace{2ex}

\begin{wrapfigure}{L}{0.3\textwidth}
  \centering
  \vspace{-3ex}
  % Ubah file gambar berikut dengan file foto dari mahasiswa
  \includegraphics[width=0.3\textwidth]{gambar/self-image.jpeg}
  \vspace{-4ex}
\end{wrapfigure}

% Ubah kalimat berikut dengan biografi dari mahasiswa
\noindent\name{}, lahir di Jogjakarta, 24 Agustus 2002 dan merupakan anak kedua dari
tiga bersaudara. Penulis menempuh pendidikan formal SD Negeri Percobaan 1
Malang, SMP Negeri 1 Malang, dan SMA Negeri 3 Malang. Penulis menjadi
salah satu mahasiswa di Departemen Teknik Informatika FTEIC - ITS pada
tahun 2021.

\cleardoublepage

\end{document}
